\chapter{Segmentación y Alineación automática}

Como la creación de recursos de manera manual es costosa, y los requerimientos de volúmen de información son altos, la generación automática o semi automática de recursos de habla toma gran relevancia en la investigación actual.

El procesamiento de grabaciones del dominio público para la creación de recursos para la investigación requiere la resolución de dos tareas concretas, que son la segmentación y la anotación definidas en el capítulo anterior.

\section{Segmentación basada en silencio}

Las segmentación basada en silencio es una respuesta natural al problema de cómo segmentar señales de voz, pues existe una correspondencia directa entre símbolos de puntuación y pausas en la voz. De igual manera en una señal de voz, los locutores requieren pausas para tomar aire, lo que permite que exista una cadencia en la señal y espacios donde realizar los cortes.

Para identificar el proceso de segmentación por silencios, es necesario entender cómo se almacenan las señales de voz digitalmente y cómo se puede procesar este tipo de datos.

La manera m\'as sencilla de representar digitalmente una señal de voz es haciendo el uso de la Modulación por Impulsos Codificados o PCM por sus siglas en inglés, donde por medio de un transductor análogo, se capturan las variaciones en la presión del aire y se registran como un rango de valores normalizado. El senso de la señal se hace periódicamente a una frecuencia determinada generando de esta manera una secuencia de valores que representa la señal. El proceso de representar la presión del aire en un segmento de tiempo se denomina cuantización, y para señales de audio se utilizan valores normalizados entre -127 y 128 representados por 8 bits a 16000 Hz de frecuencia.

En la figura \ref{img:pcm} se muestra la visualización de una grabación correspondiente a la palabra agua, extraída del corpus Open Speech Corpus de palabaras, en su longitud original y aumentada en un 500\% y 1000\% usando Audacity \cite{audacity}

\begin{figure}[H]
\caption{Representación visual de una señal de voz}
\label{img:pcm}
\includegraphics[width=\textwidth]{imagenes/04_01_pcm.png}
\end{figure}

Al hablar de silencio, nos referimos a ausencia total o parcial de sonido, lo cual se puede relacionar directamente con la intensidad de la señal en un segmento de tiempo.

Definiremos intensidad como la sumatoria de las energías en un periodo de tiempo  \cite{Jurafsky2000SpeechRecognition}    

\begin{equation}
\label{eq:energy}
I = \sum{x^2}  
\end{equation}

Esto nos daría un indicador de toda la señal, sin embargo, es útil realizar este análisis por pequeños segmentos del audio para identificar en conjunto cuáles son los puntos donde la intensidad baja representa un espacio de silencio. Estos segmentos por lo general se definen en espacios de 25ms solapados cada 10ms \cite{Jurafsky2000SpeechRecognition}, lo cual da la idea de una señal continua, donde cada segmento comparte información con el anterior. 

Utilizando esta idea es posible tomar cualquier señal de voz, segmentarla cada 25ms y encontrar los segmentos consecutivos donde la intensidad sea baja o cercana a cero y estos segmentos consecutivos representarían pausas de silencio. Algunas consideraciones a tomar al usar esta aproximación es que incluso en medio de las señales de voz, existen subsegmentos donde la intensidad es baja, por ejemplo en la pronunciación de consonantes plosivas, como la b, c, d, g, p y t existe una interrupción momentánea y completa del flujo de aire, causando momentáneamente segmentos de silencio.

Con base en las anotaciones del corpus de experimentación fonética se determinó que la duración promedio de cada consonante plosiva es inferior a los 100ms. También, extrayendo información del corpus de pruebas se determinó que los espacios de silencio son de aproximadamente 500ms. Con esta información se decidió utilizar segmentos 250ms de longitud y desplazamiento de 100ms. Se determinan como silencios los conjuntos de cinco o más segmentos consecutivos de intensidad baja.

El criterio de silencio seleccionado como baja intensidad de múltiples segmentos consecutivos plantea retos con respecto al valor concreto de energía para determinar silencio o señal. Considerando que los locutores no grabaron a un volúmen estándar y que la relación ruido señal varía en todas las grabaciones es distinta, se propusieron dos aproximaciones para determinar los segmentos de silencio, la primera normalizando los valores de la señal y definiendo un umbral fijo de intensidad en proporción al valor mas alto de la grabación, considerando valores de 0.3, 0.15 y 0.05 que representan umbrales de intensidad del 30\%, 15\%, y 5\%. También se experimentó utilizando algoritmos de agrupamiento con dos centroides iniciales en 0 y la intensidad máxima. Los resultados se presentan en la tabla \ref{tab:resultados_segmentacion_silencios}.

\begin{table}[H]
\centering
\caption{Resultados de la segmentación por silencios}
% \caption{Speech English Corpus}
\label{tab:resultados_segmentacion_silencios}
\begin{tabular}{|l|l|}
\textbf{Unbral} & \textbf{Precisión} \\ \hline
Fijo 30\%  & 57.01\% \\  \hline
Fija 15\% & 83.26\% \\  \hline
Fija 5\% & 69.05\% \\  \hline
Dinámico  & 74.53\% \\  \hline
\end{tabular}
\end{table}


\section{Alineación ingenua}

La alineación consiste en, después de tener una segmentación por silencios en las grabaciones y una segmentación por símbolos de puntuación en la transcripción, encontrar la relación que existe entre los segmentos de audio y los de tokens del texto.

La primera aproximación ingenua consiste en relacionar en orden los segmentos de voz y texto, asumiendo que los signos de puntuación tienen siempre una ocurrencia acústica representada como silencio, y por consiguiente, al segmentar el texto por signos de puntuación existe una relación uno a uno con cada segmento delimitado por silencios.

Para la evaluación de este modelo se usa el indicador WER, del inglés Word Error Rate, el cual utilizando parámetros de sustitución (S), eliminación (E) e inserción (I) de la distancia de edición Levenstein \cite{Levenshtein_SPD66} y la totalidad de palabras (N) se obtiene la relación mostrada en la ecuación \ref{eq:wer}

\begin{equation}
    \label{eq:wer}
    WER = \frac{S + E + I}{N}
\end{equation}

Si los valores esperados y calculados difieren en su totalidad, es posible que el WER tenga valores superiores a 1, lo que indica una alineación completamente equivocada.

Como el proceso de alineación busca encontrar la similaridad de las anotaciones en los segmentos, se define un Align Error Rate, compuesto de la suma ponderada de cada WER en un segmento por la cantidad de palabras esperadas en el segmento, como se muestra en la ecuación \ref{eq:aer}

% \macb{Esta m\'etrica penaliza mucho m\'as las eliminaciones, si un segmento deb\'ia tener 3 palabras, pero resultaron 6, el WER (que es al menos 3 puntos de eliminaci\'on) va a costar el doble. Sin embargo no hay una motivaci\'on clara de porqu\'e se deber\'ia penalizar m\'as el costo de eliminaci\'on. Si por el contrario solo hay una palabra cuando deber\'ian haber 3, el WER ser\'a s\'olo de 0.33. Mi sugerencia es manejar una ponderaci\'on m\'as general, ponderando la sumatoria de WER por cada segmento con el n\'umero total de palabras en todos los segmentos. Sugerencia: $AER = \frac{\sum{WER_i}}{\sum{N_i}}$}



\begin{equation}
    \label{eq:aer}
    AER = \sum_{i=0}^{n}{WER_i * \frac{W_i}{W}}
\end{equation}

Donde $n$ es el número de segmentos en la grabación, $W$ representa el número de palabras en la transcripción de la grabación, $W_i$ el numero de palabras en el respectivo segmento y $WER_i$ el WER del respectivo segmento.

Calculando el AER para la anotación ingenua se obtiene el siguiente resultado, expresado en valor porcentual.

\begin{equation}
    AER_{ingenuo} = 79.13
\end{equation}

En la figura \ref{img:aer_ingenuo} se muestra gráficamente los resultados de cada AER por grabación evaluada, donde se observa que en la mayoría de casos, los valores AER son altos, aunque en algunas grabaciones de corta duración las pausas corresponden a signos de puntuación, estos fenómenos no representa una regla.

\begin{figure}[H]
\caption{Resultados de la alineación ingénua}
% \macb{usar notaci\'on fon\'etica, por ejemplo /a/ en SAMPA, la letra A en SAMPA es una vocal del ingl\'es, igualmente con E.}
\label{img:aer_ingenuo}
\begin{center}
\includegraphics[width=0.7\textwidth]{imagenes/04_01_aer_ingenuo.png}
\end{center}
\end{figure}

%Donde se observa que al ser un valor muy alto, dando la idea que muchos segmentos se encuentran desalineados


\section{Alineación basada en duración de fonemas}

El problema con la alineación ingénua consiste en que las pausas no necesariamente corresponden siempre a un símbolo de puntuación, pues en caso de oraciones largas, el locutor debe respirar, y en oraciones muy cortas, el locutor omite la pausa para dar continuidad a la lectura.

Resaltando que existe igualmente un orden secuencial entre ambos segmentos, la aproximación basada en la duración de fonemas utiliza la duración promedio de fonemas extraída del corpus fonético de experimentación y reportada en la tabla \ref{tab:duracion_promedio_fonemas}. Con esta información y transformando las letras de cada token a su correspondiente fonema, se obtiene una duración estimada de cada token, la cual puede ser usada para alinear con mayor precisión.



\begin{table}[H]
\centering
\caption{Distribución fonética}
\label{tab:duracion_promedio_fonemas}
\begin{tabular}{|l|l|l|}
\textbf{Fonema} &\multicolumn{1}{|p{2.2cm}|}{\textbf{Duración promedio}} & \multicolumn{1}{|p{2.2cm}|}{\textbf{Desviación estándar}} \\ \hline
sil & 0.6619 & 0.3700 \\ \hline
a   & 0.1421 & 0.0566 \\ \hline
o   & 0.1487 & 0.0607 \\ \hline
e   & 0.1204 & 0.0416 \\ \hline
n   & 0.1090 & 0.0442 \\ \hline
i   & 0.1294 & 0.0404 \\ \hline
l   & 0.1125 & 0.0530 \\ \hline
s   & 0.1811 & 0.1059 \\ \hline
t   & 0.0784 & 0.0507 \\ \hline
d   & 0.0877 & 0.0536 \\ \hline
R   & 0.0999 & 0.0517 \\ \hline
b   & 0.0826 & 0.0610 \\ \hline
r   & 0.0750 & 0.0254 \\ \hline
p   & 0.0692 & 0.0649 \\ \hline
j   & 0.1038 & 0.0659 \\ \hline
u   & 0.1441 & 0.0507 \\ \hline
m   & 0.1183 & 0.0515 \\ \hline
k   & 0.1224 & 0.0904 \\ \hline
g   & 0.1016 & 0.0970 \\ \hline
f   & 0.1402 & 0.0569  \\ \hline
c   & 0.1032 & 0.0838  \\ \hline
y   & 0.1118 & 0.0044  \\ \hline
C   & 0.1274 & 0.0248  \\ \hline
N   & 0.0575 & 0  \\ \hline
S   & 0.0927 & 0  \\ \hline
\end{tabular}
\end{table}

Este problema entonces se simplifica a uno de optimización donde dados dos arreglos numéricos, donde el primero representa la duración en segundos de cada segmento y el segundo la duración estimada de cada token, se debe encontrar una relación entre los índices de cada arreglo que minimice la diferencia de segundos entre los segmentos correspondientes. 

Con dicha estrategia se obtiene un AER de 58.21\%.

En la imagen \ref{img:aer_duracion} se muestran los resultados individuales de cada grabación. De esta manera se observa un conjunto de grabaciones de baja duración, inferior a los 80 segundos en las cuales el AER es del 10.83\%, en las grabaciones con duración superior el AER es de 94.68\%, el cual representa el mismo fenómeno de la anotación ingénua, pues la elección erronea de un segmento propaga el error en los segmentos siguientes.

\begin{figure}[H]
\caption{Resultados de la alineación por duración de fonemas}
% \macb{usar notaci\'on fon\'etica, por ejemplo /a/ en SAMPA, la letra A en SAMPA es una vocal del ingl\'es, igualmente con E.}
\label{img:aer_duracion}
\begin{center}
\includegraphics[width=0.7\textwidth]{imagenes/04_03_aer_duracion_fonemas.png}
\end{center}
\end{figure}

%\begin{equation}
%    AER_{duracion fonemas} = 12.5
%\end{equation}

\section{Segmentación basada en información acústica}

Las grabaciones voz no son señales estacionarias, pero al segmentar en pequeños trozos de tamaño inferior al de la duración de los fonemas, estos segmentos cuando pertenecen a vocales y consonantes sonoras son estacionarios.

Si bien las características morfológicas de los locutores modifican su timbre y tono de voz, a nivel acústico, la dicción de cada fonema es similar. En las vocales, que son los fonemas con más energía, las partes de la anatomía responsables de diferenciar los sonidos son la apertura de la glotis, la posici\'on de la lengua y la apertura de la boca. Lo cual da la idea que una descomposición acústica podría identificar este fenómeno.

Al descomponer cualquier señal utilizando la transformada de Fourier (ver equación \ref{eq:fourier}), se obtienen señales raíz de cada onda compleja. Con las señales de voz, los coeficientes máximos  en el orden ascendente de la frecuencia se denominan formantes, y el formante 1 (F1) y el formante 2 (F2) representan efectivamente el fenómeno efectuado por las cuerdas vocales vibrando y la resonancia dada por la cavidad bucal. Los formantes son harmónicos, la captura de estos se da en señales de cualquier tipo, sea de voz u otro fenómeno acústico, como las señales generadas por instrumentos musicales. Todos los formantes F1,F2,F3, etc. corresponden al pulso excitativo de la glotis y la resonancia en el tracto vocal.

\begin{equation}
\label{eq:fourier}    
f(x) = \frac{1}{2} \, a_{0} + \sum_{n=1}^{\infty} \left[
   a_{n}\,\mathbf{cos} (n\,x) + b_{n} \,\mathbf{sin} (n\,x) \right]
\end{equation}

Los formantes teóricos para las cinco vocales del español se muestran en la tabla \ref{tab:formantes_teoricos} tomada de \cite{Bradlow1995}. Dichos formantes fueron calculados promediando la medición de formantes F1 y F2 de varios locutores con dialecto peninsular.

\begin{table}[H]
\centering
\caption{Formantes teóricos para vocales del español \cite{Bradlow1995}}
\label{tab:formantes_teoricos}
\begin{tabular}{|l|l|l|l|l|}
\hline
\textbf{Vocal} & \textbf{f1} & \textbf{f2} & \textbf{std f1} & \textbf{std f2} \\ \hline
a   & 638 & 36 & 1353 & 84 \\ \hline
e   & 458 & 42 & 1800 & 131 \\ \hline
i   & 286 & 6  & 2200 & 131 \\ \hline
o   & 460 & 19 & 1019 & 99 \\ \hline
u   & 300 & 20 & 992  & 121 \\ \hline

\end{tabular}
\end{table}

También se realizó una observación de las vocales calculando sus formantes basados en la información extra\'ida de las palabras anotadas fonéticamente. Esta información se encuentra en la tabla \ref{tab:formantes_observador}. Los formantes fueron calculados con Parselmouth \cite{parselmouth}.

\begin{table}[H]
\centering
\caption{Formantes observados para vocales del español}
\label{tab:formantes_observador}
\begin{tabular}{|l|l|l|l|l|}
\hline
\textbf{Vocal} & \textbf{f1} & \textbf{f2} & \textbf{std f1} & \textbf{std f2} \\ \hline
a   & 682.84 & 1347.66 & 73.77 & 93.04 \\ \hline
e   & 494.84 & 1619.97 & 55.49 & 132.84\\ \hline
i   & 382.33 & 1657.33 & 50.65 & 155.24 \\ \hline
o   & 506.24 & 1101.14 & 60.20 & 170.64\\ \hline
u   & 434.77 & 984.185 & 50.32 & 193.66\\ \hline

\end{tabular}
\end{table}

A partir de esta información de la tabla \ref{tab:formantes_observador} se realizaron experimentos buscando la precisión en la identificación de las vocales. Para ello se tomó la norma L1 \cite{manhatattan_distance} de los dos primeros formantes, definiendo rango de aceptación, entre los 50Hz y 500Hz; y utilizando la desviación estándar de cada vocal como rango de aceptación. Adicionalmente a esto, se entrenó un algoritmo de agrupación no supervizado, K-means, inicializando los centroides en los formantes observados en el corpus de experimentación fonética.

Estos experimentos se pusieron a prueba en todo el corpus de experimentación fonética, de donde se extrajeron los segmentos correspondientes a vocales y se evaluó la precisión de cada aproximación.

Los resultados se reportan en la tabla \ref{tab:resultados_vocales}.

\begin{table}[H]
\centering
\caption{Precisión identificación vocálica}
\label{tab:resultados_vocales}
\begin{tabular}{|l|l|l|l|l|}
\hline
Aproximación              & Precisión & Recall & F1 Score \\ \hline
Teórica mas 50 Hz         & 44.36     & 45.51  & 44.93    \\ \hline
Teórica mas 100 Hz        & 46.29     & 47.49  & 46.88    \\ \hline
Teórica mas 200 Hz        & \textbf{50.36}     & 51.66  & 51.00    \\ \hline
Teórica mas 300 Hz        & 47.71     & 48.95  & 48.32    \\ \hline
Teórica mas 400 Hz        & 40.75     & 41.80  & 41.27    \\ \hline
Teórica mas 500 Hz        & 36.81     & 37.76  & 37.28    \\ \hline
Observados mas desviación & 41.76     & 42.84  & 42.29    \\ \hline
K-means mas 200           & 45.86     & 47.05  & 4645     \\ \hline
\end{tabular}

\end{table}

Para entender los bajos resultados presentados por la aproximación basada en los formantes observados y su desviación (50.36\% de precisión), se realizó un análisis visual sobre grabaciones con las vocales que se extrajeron del Open Speech Corpus. Se graficó simultaneamente la onda, los formantes F1 y F2 donde la intensidad superaba los 60 decibeles, es decir, $I > 60 dB$, y los formantes F1 y F2 promedio y sus desviaciones observados en el corpus. Las figuras \ref{img:formantes_a}, \ref{img:formantes_e}, \ref{img:formantes_i}, \ref{img:formantes_o} y \ref{img:formantes_u} muestran ejemplos de dicho análisis. Los puntos indican los formantes calculados, F1 en azul y F2 en rojo, las líneas azul y roja representan los valores promedios para los formantes 1 y 2 respectivamente, con una franja del mismo color y en leve transparencia representando los rangos aceptados para cada formante considerando la desviación estándar, en verde la señal, y en amarillo la intensidad.

% (Y explicar las lineas rojas, amarilla, verde. Recomiendo que los formantes observados sean graficados con la desviaci\'on st\'andar en transparencia. Tambi\'en, los dos formantes observados deber\'ian tener color distinto entre ellos y posiblemente en la misma escala de los calculados.)
%\ref{img:formantes_a} y e en la figura \ref{img:formantes_e}

% Ejemplo de cómo graficar la desviación estándar: https://i.stack.imgur.com/pDYIh.png


\begin{figure}[H]
\caption{Formantes de vocal A}
% \macb{usar notaci\'on fon\'etica, por ejemplo /a/ en SAMPA, la letra A en SAMPA es una vocal del ingl\'es, igualmente con E.}
\label{img:formantes_a}
\begin{center}
\includegraphics[width=0.7\textwidth]{imagenes/04_02_a.png}
\end{center}
\end{figure}

\begin{figure}[H]
\caption{Formantes de vocal E}
\label{img:formantes_e}
\begin{center}
\includegraphics[width=0.6\textwidth]{imagenes/04_02_e.png}
\end{center}
\end{figure}

\begin{figure}[H]
\caption{Formantes de vocal I}
\label{img:formantes_i}
\begin{center}
\includegraphics[width=0.6\textwidth]{imagenes/04_02_i.png}
\end{center}
\end{figure}

\begin{figure}[H]
\caption{Formantes de vocal O}
\label{img:formantes_o}
\begin{center}
\includegraphics[width=0.6\textwidth]{imagenes/04_02_o.png}
\end{center}
\end{figure}

\begin{figure}[H]
\caption{Formantes de vocal U}
\label{img:formantes_u}
\begin{center}
\includegraphics[width=0.7\textwidth]{imagenes/04_02_u.png}
\end{center}
\end{figure}

% Esto debería estar arriba.
% En ambas figuras se muestra en verde la onda, en azul el formante 1 en rojo el formante dos, en amarillo la intensidad cuyo valor se encuentra en el eje derecho y en rojo y azul claro los rangos esperados siendo la línea centra el formante observado promedio y las lineas adyacentes los valores de sumar y restar la desviación estándar.

En las gráficas se observa que en los rangos donde hay mas energía, los formantes se estabilizan cerca de ciertos valores, sin embargo no siempre el valor de estabilización corresponde a los rangos cercanos al formante promedio más o menos su desviación estándar.

Considerando que en los valores de los formantes y su rango de aceptación no genera los resultados esperados, se realizan experimentos con árboles de decisión. Para estos experimentos, se tomaron los mismos valores de los formantes y la intensidad usando también valores de los formantes 3, 4 y 5, que enriquecen la información acústica extraída de las grabaciones.

Los conjuntos de entrenamiento y pruebas se extrajeron del corpus de experimentación fonética, realizando una partición de 90\% para entrenamiento y 10\% para pruebas. Esta segmentación se realizó creando ventanas de tiempo de 60ms. Se contó la ocurrencia de cada fonema en las ventanas y sobre el total de apariciones de fonemas en las ventanas se determinó la proporción 90\%-10\% para los conjuntos de entrenamiento y pruebas.

Como los árboles de decisión son sensibles a la cantidad de grupos a clasificar, se realizaron experimentos transformando la anotación fonética de cada grabación. Las transformaciones usadas se muestran en la tabla \ref{tab:transformacion_fonetica_arboles}.

\begin{table}[H]
\centering
\caption{Clasificaci\'on fonética para modelos de \'arboles de decisiones.}
\label{tab:transformacion_fonetica_arboles}
\begin{tabular}{|l|l|l|l|l|}
\hline
\textbf{Fonema} & \multicolumn{1}{|p{2.5cm}|}{\textbf{Vocales / No Vocales}} &  \multicolumn{1}{|p{3cm}|}{\textbf{Consonantes sonoras / No sonoras}} &  \multicolumn{1}{|p{2.5cm}|}{\textbf{Tipos de articulación}} & \multicolumn{1}{|p{2.3cm}|}{\textbf{Vocales Extremas}} \\\hline
a   & a         & a          & a           & a         \\ \hline
o   & o         & o          & o           & no\_vocal \\ \hline
e   & e         & e          & e           & no\_vocal \\ \hline
n   & no\_vocal & sonora     & nasal       & no\_vocal \\ \hline
i   & i         & i          & i           & i         \\ \hline
l   & no\_vocal & sonora     & aproximante & no\_vocal \\ \hline
s   & no\_vocal & sonora     & fricativa   & no\_vocal \\ \hline
t   & no\_vocal & no\_sonora & plosiva     & no\_vocal \\ \hline
d   & no\_vocal & no\_sonora & plosiva     & no\_vocal \\ \hline
R   & no\_vocal & sonora     & vibrante    & no\_vocal \\ \hline
b   & no\_vocal & no\_sonora & plosiva     & no\_vocal \\ \hline
r   & no\_vocal & no\_sonora & plosiva     & no\_vocal \\ \hline
p   & no\_vocal & no\_sonora & plosiva     & no\_vocal \\ \hline
j   & no\_vocal & sonora     & aproximante & no\_vocal \\ \hline
u   & u         & u          & u           & u         \\ \hline
m   & no\_vocal & sonora     & nasal       & no\_vocal \\ \hline
k   & no\_vocal & no\_sonora & plosiva     & no\_vocal \\ \hline
g   & no\_vocal & sonora     & plosiva     & no\_vocal \\ \hline
f   & no\_vocal & sonora     & fricativa   & no\_vocal \\ \hline
y   & no\_vocal & sonora     & fricativa   & no\_vocal \\ \hline
C   & no\_vocal & sonora     & fricativa   & no\_vocal \\ \hline
N   & no\_vocal & sonora     & nasal       & no\_vocal \\ \hline
S   & no\_vocal & sonora     & fricativa   & no\_vocal \\ \hline
\end{tabular}
\end{table}

Los resultados de los experimentos se muestran a continuación en la tabla \ref{tab:resultados_arboles}

\begin{table}[H]
\centering
\caption{Resultados de la clasificación fonética usando árboles de decisión}
\label{tab:resultados_arboles}
\begin{tabular}{|l|l|l|l|}
\hline
Aproximación                     & Precisión & Recall & F1 Score \\ \hline
Todos los fonemas                & 42.34     & 42.34  & 42.34    \\ \hline
Tipos de articulación            & 50.76     & 52.30  & 52.30    \\ \hline
Consonantes sonoras / No sonoras & 62.86     & 62.86  & 62.86    \\ \hline
Vocales \ No Vocales             & 69.62     & 69.62  & 69.62    \\ \hline
Vocales extremas                 & \textbf{81.21}     & \textbf{81.21}  & \textbf{81.21}    \\ \hline


\end{tabular}
\end{table}

Utilizando el clasificador de vocales extremas, se realizó el cálculo del AER, haciendo una transformación de los tokens al diccionario fonético y posteriormente evaluando el WER del texto esperado con respecto al predicho por el clasificador.

En la imagen \ref{img:decision_tree} se visualiza el árbol de decisión usado para obtener los resultados del experimento, donde se tiene un árbol de 7 niveles, las características usadas para la separación de este árbol fueron las 5 primeras formantes y la intensidad.

\begin{figure}[H]
\caption{Árbol de decisión de vocales externas}
\label{img:decision_tree}
\begin{center}
\includegraphics[width=1.1\textwidth,angle=90]{imagenes/external_vowels.png}
\end{center}
\end{figure}

Como el clasificador retorna un posible fonema por cada segmento de duración de 60ms, se comprimió el resultado predicho agrupando todos los fonemas adyacentes en uno solo. Considerando de igual manera que la duración promedio de las vocales evaluadas es cercana a los 150ms con desviación de 50ms, los grupos con menos de 4 fonemas adyacentes se eliminan, pues no representan ninguna vocal.

Realizando estas adaptaciones, el valor de AER de la anotación usando información fonética fue: 89.56\% y los resultados individuales se muestran en la figura \ref{img:aer_vocales_externas}.

\begin{figure}[H]
\caption{Resultados de la alineación usando información fonética}
\label{img:aer_vocales_externas}
\begin{center}
\includegraphics[width=0.7\textwidth]{imagenes/04_04_aer_vocales_externas.png}
\end{center}
\end{figure}

El principal problema con esta aproximación es que, al transformar el texto a esta representación fonética, toda la información de las consonantes y dos vocales se pierde, reduciendo el tamaño del texto y haciendo que cada error le dé un valor más alto a cada WER local. En algunos casos, al finalizar la transformación del texto, se encontraron escenarios donde en un token no existía ninguna vocal externa, dejando este segmento de varios segundos como un solo fonema. Adicional a esto, la propagación de los errores a segmentos adyacentes sigue presente.

\section{Comparación de técnicas}

En este capítulo de discutieron tres aproximaciones para realizar una anotación automática de corpus extraídos de audiolibros, usando una segmentación automática basada en silencios y un umbral de ruido del 15\%, el cual da una precisión del 83.26\% la cual hace que sea bastante atractiva como herramienta de segmentación en comparación con una segmentación manual.

Con respecto a la anotación, los mejores resultados se obtuvieron a partir de la segmentación basada en duración promedio de fonemas. Para el caso puntual de los audiolibros donde la lectura es continua y hay una cadencia constante, esta técnica obtuvo buenos resultados, sin embargo, en el corpus escogido, existen distintos tipos de textos, y tipos de lectura, y textos donde el locutor enfatice mas las pausas, como en los textos líricos, esta aproximación basada en duración promedio de fonemas empeora los resultados. En casos, donde los textos son cortos y los locutores no tienen espacios para variar su cadencia, esta técnica muestra mejores resultados

La anotación basada en fonemas, aunque presenta una alta precisión a la hora de identificar características aisladas, no puede transferir esta precisión a la anotación del texto completo. Usando el clasificador con mejores resultados, que contempla solo vocales externas y no vocales, existen pequeños espacios de tiempo, donde las características no son estables y se genera un cambio de unidad fonética, que afecta los resultados. De igual manera, palabras que incluyen fonemas adyacentes con la misma categoría, como consonantes adyacentes o hiatos, deforman las palabras a un punto donde no hay información suficiente para determinar los límites en el tiempo.

Con estos resultados, el corpus generado por la técnica basada en fonemas, que tuvo un mejor AER, presentaba muchos errores. Realizando verificaciones manuales de los segmentos generados para grabaciones de mas de 100 segundos, se observaron desalineaciones que se propagaban hasta el final del texto, haciendo que los resultados no fueran apropiados para su publicación, pues el error es muy alto.

En comparación con herramientas existentes y abiertas, las técnicas exploradas en este trabajo no ofrecen ninguna ventaja. 

Es válido aclarar que muchas de las herramientas en el estado del arte utilizan modelos acústicos pre-generados y que son aveces usados en tareas de reconocimiento de voz. Es usual encontrar modelos que fueron generados por corpus que no son de licencia abierta, o no están disponibles, dando un poco de ventaja a la herramienta. Existen también herramientas de código abierto que usan herramientas pagas de terceros, y reportan mejores resultados. 