\chapter{Conclusiones}

El presente trabajo exploró las tareas de segmentación y alineación del procesamiento de voz, reconociendo proponiendo aproximaciones computacionales apoyadas en el procesamiento digital de señales, la fonología y las ciencias de la computación para su resolución.

En los últimos tres años el incremento de recursos de licencia abierta en español para el procesamiento de voz ha aumentado considerablemente, mostrando la relevancia de estos recursos en la investigación.

Aproximaciones automáticas y colaborativas se han propuesto con éxito para la creación de recursos de voz.

Fuentes de datos de voz abiertas, como audio libros y grabaciones de sesiones parlamentarias, han sido usadas para la creación de corpus de voz.