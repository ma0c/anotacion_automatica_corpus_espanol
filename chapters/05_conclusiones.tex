\chapter{Conclusiones}

El presente trabajo exploró las tareas de segmentación y alineación del procesamiento de voz, reconociendo proponiendo aproximaciones computacionales apoyadas en el procesamiento digital de señales, la fonología y las ciencias de la computación para su resolución.

En los últimos tres años el incremento de recursos de licencia abierta en español para el procesamiento de voz ha aumentado considerablemente, mostrando la relevancia de estos recursos en la investigación.

Aproximaciones automáticas y colaborativas se han propuesto con éxito para la creación de recursos de voz.

Fuentes de datos de voz abiertas, como audio libros y grabaciones de sesiones parlamentarias, han sido usadas para la creación de corpus de voz.

Los recursos anotados fonéticamente permiten un análisis detallado de la información acústica de las grabaciones para obtener información relevante en el análisis de grabaciones de voz.

La creación manual de recursos de habla es costosa y requiere tiempo de personal entrenado para la garantizar la calidad de los recursos.

El uso de recursos existentes acelera el desarrollo de nuevos recursos, usando los primeros como insumo para la creación de más recursos.

La segmentación de grabaciones de voz usando análisis de frecuencia en cortos periodos de tiempo y calculando la energía de cada segmento y definiendo un umbral mínimo con respecto a la energía máxima representa las pausas naturales en las grabaciones de voz

Usando información fonética y acústica comunes a un lenguaje es posible obtener resultados satisfactorios para las tareas de alineación, sin la necesidad de usar algoritmos que requieren grandes volúmenes de datos.
