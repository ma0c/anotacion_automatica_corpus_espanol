\chapter{Introducción}

El procesamiento de voz es un área de investigación activa que \macb{\st{pertenece al procesamiento} combina aspectos del procesamiento de se\~nales y estudio} del lenguaje natural. Entre las tareas destacadas de esta área de investigación destacan la síntesis y el reconocimiento de voz, que han sido un problema abordado desde hace más de un siglo, con aproximaciones diversas que van desde el procesamiento de señales hasta las redes neuronales profundas. \macb{\st{A lo largo de todas las} En las \'ultimas décadas de investigación en este campo\st{, se ha mostrado que las mejores aproximaciones para la resoluci\'on de tareas requieren un gran volumen de datos, el nombre que se le da a estas recopilaciones de datos es corpus.} se han logrado mejores resultados en diversas tareas como la s\'intesis de voz haciendo uso de gran volumen de datos.}

Los corpus \macb{o colleci\'on de datos} deben cumplir con características para poder ser usados en investigación, entre las que se encuentran género, dialecto y edad del locutor, las condiciones de grabación, ruido de ambiente, origen de la grabación, el lenguaje usado, y el tipo de distribución.

La creación de corpus para abordar las distintas tareas en el procesamiento de voz es por lo general costosa, pues incluye la consecución de locutores, los recursos de hardware para realizar las grabaciones en condiciones apropiadas y el personal para el enriquecimiento de los recursos, como la segmentación, anotación y clasificación. Dado que este proceso es costoso, muchos recursos se distribuyen bajo licencias cerradas, lo que dificulta el acceso de los mismos.

Este trabajo busca la creación de un corpus usando grabaciones previamente recolectadas y distribuidas bajo licencia abierta, utilizando mecanismos automáticos para la segmentación y anotación de los mismos.

\section{Definición del problema}

Los recursos para la investigación de tecnologías del habla son de vital importancia para la generación y validación de nuevo conocimiento en el área. En la actualidad donde la investigación se ha decantado en su mayoría por el aprendizaje de máquina utilizando mecanismos de aprendizaje supervisado, semi supervisado y no supervisado \cite{Chiu2018} \cite{AmazonSemiSupervised}  \cite{ZeroResources}, la calidad y tamaño de los corpus usados para el entrenamiento, validación y pruebas definen en gran medida los resultados de la investigación \cite{Hernandez-Mena2017AutomaticResources}.


Como la creación de corpus de es costosa \cite{googleTTSLatinAmericanSpanishCorpus}, se propone utilizar grabaciones distribuidas bajo licencia libre y realizar el procesamiento requerido para formar un corpus para el procesamiento de voz.

Se usarán las grabaciones del proyecto Librivox \cite{LibriVox}, una iniciativa que, por medio de voluntarios, graba audio libros que pertenecen al dominio público. En su mayoría, los audio libros se encuentran publicados en el proyecto Gutenberg \cite{gutenberg}. Cada libro es grabado por capítulo y cada capítulo es grabado por un solo locutor.

Dado que las grabaciones tienen una longitud de varios minutos, se realizará un proceso de segmentación donde a partir de cada grabación se obtienen trozos mas pequeños de tamaño conforme a otros corpus. Posteriormente se realizará un proceso de alineación sobre estos segmentos, donde se determinará a partir de la información acústica de cada segmento de grabación, la declaración correspondiente.


\section{Justificación}

En la literatura se encuentran recursos representativos reconocidos para realizar tareas comunes como el reconocimiento automático del habla y la síntesis del habla como, TIMIT \cite{PriceTheRecognition}, Darpa Resource Management \cite{Lucke1992ExpandingCorpus}, Switchboard \cite{Godfrey1992SWITCHBOARD:Development}, Call Home  \cite{Fu-HuaLiuSpeechCorpus},  Cambridge Read News \cite{RobinsonWSJCAM0:RECOGNITION}, Aurora Digits \cite{EvansEfficientCorpus}, Fisher \cite{CieriTheSpeech-to-Text} y Libri Speech \cite{LIBRISPEECH}. Estos corpus se anotados a varios niveles, desde el nivel fonético, donde se describe los momentos en el tiempo donde inicia y termina un fonema; hasta el nivel de declaración donde s\'olo se relaciona una frase típicamente corta con su correspondiente grabación. Todos estos recursos son de lengua inglesa. En la tabla \ref{tab:english_corpora} se extendiende la información de estos recursos en inglés.

\begin{table}[H]
\centering
\caption{Corpus hablados en la lengua inglesa\macb{. La duraci\'on est\'a en horas.}}
% \caption{Speech English Corpus}
\label{tab:english_corpora}
\begin{tabular}{|l|l|l|l|l|}
\textbf{Base de datos} & \textbf{Duración} & \textbf{Vocabulario} & \textbf{Anotación} & \textbf{Licencia}\\
\hline
FISHER\cite{CieriTheSpeech-to-Text}  & más de 2000 & Muy grande &  Declaración & LDC Non-Members\\
\hline
LIBRISPEECH\cite{PanayotovLIBRISPEECH:BOOKS}  & más de 1000 & Muy grande &  Declaración & CC BY 4.0\\
\hline

Switchboard \cite{Godfrey1992SWITCHBOARD:Development}  & 240 & Muy grande & Palabras & LDC Non-Members\\
\hline
\multicolumn{1}{|p{3.4cm}|}{DARPA Resource Management \cite{Lucke1992ExpandingCorpus}} & No especificado  & Grande & Declaración & LDC Non-Members \\
\hline
Cambridge Read News \cite{RobinsonWSJCAM0:RECOGNITION}  & No especificado & Muy grande & Fonético & LDC Non-Members\\
\hline
Call Home  \cite{Fu-HuaLiuSpeechCorpus} & 18.3 & Grande & Declaración & LDC Non-Members\\
\hline
TIMIT \cite{PriceTheRecognition} & 5.4 & Grande & Fonético  & LDC Non-Members\\
\hline
Aurora Digits \cite{EvansEfficientCorpus}  & Menos de 1 & Pequeño & Palabra & ELRA\\
\hline



\end{tabular}
\end{table}

Existen recursos para el español como Call Friend \cite{CALLFRIENDSpa}, Call Home \cite{CALLHOMESpa}, DIMEx100 \cite{Pineda2004DIMEx100:Spanish}, Heróico \cite{heroico}, Voxforfe \cite{Voxforge.org}, Fisher Spanish \cite{FischerSpa}, Albazín \cite{CampilloAlbayzinEvaluation} y CIEMPIESS \cite{Hernandez-MenaCIEMPIESS:Corpus}, los cuales se muestran en la tabla \ref{tab:spanish_corpora}, sin embargo \macb{\st{estos recursos} al comparar\st{se con los} estos recursos con aquellos en\st{de}} inglés se evidencia una diferencia en duración considerable.

\begin{table}[H]
\centering
\caption{Corpus hablados en la lengua española}
\label{tab:spanish_corpora}
\begin{tabular}{|l|l|l|l|l|}
\textbf{Base de datos} & \textbf{Duración (h)}  & \textbf{Anotación} & Licencia\\
\hline
Fisher Spanish \cite{FischerSpa}  & 163  & Declaración  & LDC Non-Members\\
\hline
CALL FRIEND Spanish \cite{CALLFRIENDSpa}  & 85  & Declaración & LDC Non-Members\\
\hline
CALL HOME Spanish \cite{CALLHOMESpa}  & 70  & Declaración & LDC Non-Members\\
\hline
Voxforge \cite{Voxforge.org}  & 51 & Declaración & GPL\\
\hline
CIEMPIESS\cite{Hernandez-MenaCIEMPIESS:Corpus}  & 17  & Declaración & CC BY-SA 4.0\\
\hline
Heroico \cite{HeroicoCorpus}  & 13 & Declaración & LDC Non-Members  \\
\hline
DIMEx100\cite{Pineda2004DIMEx100:Spanish}  & 5.6  & Fonético & \multicolumn{1}{|p{4cm}|}{Gratis para propósitos académicos}\\
\hline
Albayzín\cite{CampilloAlbayzinEvaluation}  & No especificado  & Declaración & ELRA\\
\hline
\end{tabular}
\end{table}

Se observa en ambas tablas que la mayoría de recursos son de licencia cerrada, sin embargo se destaca el corpus Libri Speech \cite{LIBRISPEECH}, el cual recopila mas de 1000 horas y está distribuido bajo la licencia Creative Commons versión 4. Este corpus se ha convertido en un recurso usado como línea base en múltiples investigaciones \cite{libribox_benchmark1,librilight,libribox_benchmark3}. 

La presencia de este corpus, validada por investigaciones posteriores abre la puerta a desarrollos similares para otros lenguajes utilizando la misma fuente de datos pública Libri Vox. 

% Aunque  existen notables diferencias entre los corpus del español y del inglés, el lenguaje español no se considera un lenguaje con escasos recursos \macb{\st{, pues e}. E}xisten múltiples corpus, bancos de árboles, datos anotados y transcritos, diccionarios y gramáticas formales \cite{CavarGlobalGORILLA}. 


\section{Objetivos}


\subsection{Objetivo general}

Implementar un alineador forzado con el fin de generar automáticamente un corpus para la lengua española de gran vocabulario y larga duración a partir de recursos existentes

\subsection{Objetivos específicos}

\begin{enumerate}
    \item Estudiar los algoritmos para alineadores forzados y sus implementaciones en el estado del arte de forma que soporten la implementación de uno para la lengua española.
    \item Recolectar recursos hablados existentes y de licencia abierta existentes para la lengua española.
    \item Construir un corpus de prueba para medir el desempeño de alineadores forzados en Español.
    \item Implementar un alineador forzado que anote a nivel de palabras un corpus de gran vocabulario y larga duración para la lengua española.
    \item Diseñar de manera empírica un conjunto de pruebas con el fin de establecer el nivel de anotación del alineador

\end{enumerate}
% MACB cambié distribución por estructura y agregué frase de las conclusiones.
\section{Estructura del documento}

En el capítulo 2 se muestra una recolección de los recursos existentes y abiertos para el español, realizando un análisis detallado de cada uno de ellos.

En el capítulo 3 se diseñan los recursos de prueba para evaluar el rendimiento del alineador forzado propuesto.

El capítulo 4 muestra múltiples experimentos para los procesos de segmentación y alineación sobre las grabaciones de Libri Vox.

Finalmente, en el capítulo 5 se concluye el proyecto y se discuten posibles trabajos futuros.