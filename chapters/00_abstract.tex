\begin{abstract}
    Los recursos abiertos para el procesamiento, reconocimiento, generación y demás tareas relacionadas con las tecnologías del lenguaje hablado son parte fundamental del proceso académico e investigativo. Estos recursos deben contener características muy específicas para la ejecución adecuada de la tarea relacionada, donde los niveles de anotación, duración de las grabaciones, variedad de locutores, balance de género, tamaño del vocabulario, relación de señal/ruido y conjunto de datos para pruebas son fundamentales. A pesar de existir diversos recursos disponibles para realizar las tareas mencionadas, los recursos abiertos para trabajar las tecnologías del habla en la lengua española son escasos y en muchos casos insuficientes para realizar comparaciones con investigaciones del estado del arte propuestas para otros lenguajes. Este trabajo realiza una exploración de los recursos abiertos para el español y propone mecanismos para segmentar audio libros publicados por voluntarios bajo licencias abiertas en recursos apropiados para el procesamiento de voz.
\end{abstract}