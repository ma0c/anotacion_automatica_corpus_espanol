\documentclass[a4paper,12pt,twoside]{report}
\usepackage[left=2cm,right=2cm,top=2cm,bottom=3cm]{geometry}
\usepackage[spanish]{babel}
\usepackage[utf8]{inputenc}
\usepackage{svg}
\usepackage{xcolor}
\usepackage{graphicx}
\usepackage{listingsutf8}
\lstset{literate={ñ}{{\~n}}1}
\usepackage{float}
\usepackage{tipa}
\usepackage{lscape} 



\usepackage{xcolor}
\usepackage{soul}
\newcommand{\macb}{\textcolor{red}}

\graphicspath{ {./imagenes/} }

\include{thesis.preamble}

\title{Anotación automática de un corpus hablado de licencia abierta y larga duración para el lenguaje Español}
\author{Mauricio Collazos}
\date{March 2021}

\begin{document}

\maketitle

\begin{abstract}
    Los recursos abiertos para el procesamiento, reconocimiento, generación y demás tareas relacionadas con las tecnologías del lenguaje hablado son parte fundamental del proceso académico e investigativo. Estos recursos deben contener características muy específicas para la ejecución adecuada de la tarea relacionada, donde los niveles de anotación, duración de las grabaciones, variedad de locutores, balance de género, tamaño del vocabulario, relación de señal/ruido y conjunto de datos para pruebas son fundamentales. A pesar de existir diversos recursos disponibles para realizar las tareas mencionadas, los recursos abiertos para trabajar las tecnologías del habla en la lengua española son escasos y en muchos casos insuficientes para realizar comparaciones con investigaciones del estado del arte propuestas para otros lenguajes. Este trabajo realiza una exploración de los recursos abiertos para el español y propone mecanismos para segmentar audio libros publicados por voluntarios bajo licencias abiertas en recursos apropiados para el procesamiento de voz.
\end{abstract}

\body

\chapter{Introducción}

El procesamiento de voz es un área de investigación activa que pertenece al procesamiento del lenguaje natural. Entre las tareas destacadas de esta área de investigación destacan la síntesis y el reconocimiento de voz, que han sido un problema abordado desde hace más de un siglo, con aproximaciones diversas que van desde el procesamiento de señales hasta las redes neuronales profundas. A lo largo de todas las décadas de investigación en este campo, se ha mostrado que las mejores aproximaciones para la resolución de tareas requieren un gran volumen de datos, el nombre que se le da a estas recopilaciones de datos es corpus. 

Los corpus deben cumplir con características para poder ser usados en investigación, entre las que se encuentran género, dialecto y edad del locutor, las condiciones de grabación, ruido de ambiente, origen de la grabación, el lenguaje usado, y el tipo de distribución.

La creación de corpus para abordar las distintas tareas en el procesamiento de voz es por lo general costosa, pues incluye la consecución de locutores, los recursos de hardware para realizar las grabaciones en condiciones apropiadas y el personal para el enriquecimiento de los recursos, como la segmentación, anotación y clasificación. Dado que este proceso es costoso, muchos recursos se distribuyen bajo licencias cerradas, lo que dificulta el acceso de los mismos.

Este trabajo busca la creación de un corpus usando grabaciones previamente recolectadas y distribuidas bajo licencia abierta, utilizando mecanismos automáticos para la segmentación y anotación de los mismos.

\section{Definición del problema}

Los recursos para la investigación de tecnologías del habla son de vital importancia para la generación y validación de nuevo conocimiento en el área. En la actualidad donde la investigación se ha decantado en su mayoría por el aprendizaje de máquina utilizando mecanismos de aprendizaje supervisado, semi supervisado y no supervisado \cite{Chiu2018} \cite{AmazonSemiSupervised}  \cite{ZeroResources}, la calidad y tamaño de los corpus usados para el entrenamiento, validación y pruebas definen en gran medida los resultados de la investigación \cite{Hernandez-Mena2017AutomaticResources}.


Como la creación de corpus de es costosa \cite{googleTTSLatinAmericanSpanishCorpus}, se propone utilizar grabaciones distribuidas bajo licencia libre y realizar el procesamiento requerido para formar un corpus para el procesamiento de voz.

Se usarán las grabaciones del proyecto Librivox \cite{LibriVox}, una iniciativa que, por medio de voluntarios, graba audio libros que pertenecen al dominio público. En su mayoría, los audio libros se encuentran publicados en el proyecto Gutenberg \cite{gutenberg}. Cada libro es grabado por capítulo y cada capítulo es grabado por un solo locutor.

Dado que las grabaciones tienen una longitud de varios minutos, se realizará un proceso de segmentación donde a partir de cada grabación se obtienen trozos mas pequeños de tamaño conforme a otros corpus. Posteriormente se realizará un proceso de alineación sobre estos segmentos, donde se determinará a partir de la información acústica de cada segmento de grabación, la declaración correspondiente.


\section{Justificación}

En la literatura se encuentran recursos representativos reconocidos para realizar tareas comunes como el reconocimiento automático del habla y la síntesis del habla como, TIMIT \cite{PriceTheRecognition}, Darpa Resource Management \cite{Lucke1992ExpandingCorpus}, Switchboard \cite{Godfrey1992SWITCHBOARD:Development}, Call Home  \cite{Fu-HuaLiuSpeechCorpus},  Cambridge Read News \cite{RobinsonWSJCAM0:RECOGNITION}, Aurora Digits \cite{EvansEfficientCorpus}, Fisher \cite{CieriTheSpeech-to-Text} y Libri Speech \cite{LIBRISPEECH}. Estos corpus se anotados a varios niveles, desde el nivel fonético, donde se describe los momentos en el tiempo donde inicia y termina un fonema; hasta el nivel de declaración donde solo se relaciona una frase, típicamente corta con su correspondiente grabación. Todos estos recursos son de lengua inglesa. En la tabla \ref{tab:english_corpora} se extendiende la información de estos recursos en inglés.

\begin{table}[H]
\centering
\caption{Corpus hablados en la lengua inglesa}
% \caption{Speech English Corpus}
\label{tab:english_corpora}
\begin{tabular}{|l|l|l|l|l|}
\hline
\textbf{Base de datos} & \textbf{Duración (h)} & \textbf{Anotación} & \textbf{Licencia} & \textbf{Publicación} \\
\hline
TIMIT & 5.4  & Fonético  & LDC Non-Members & 1988\\
\hline
\multicolumn{1}{|p{4cm}|}{DARPA Resource Management } & No especificado   & Declaración & LDC Non-Members & 1992 \\ 

\hline

Switchboard & 240  & Palabras & LDC Non-Members & 1992 \\
\hline
Call Home  & 18.3  & Declaración & LDC Non-Members & 1996 \\
\hline
Cambridge Read News & No especificado  & Fonético & LDC Non-Members & 2002\\
\hline
Aurora Digits   & Menos de 1 & Palabra & ELRA & 2002 \\
\hline
FISHER  & más de 2000 &  Declaración & LDC Non-Members & 2004\\
\hline
LIBRISPEECH  & más de 1000  &  Declaración & CC BY 4.0 & 2014 \\

\hline



\end{tabular}
\end{table}

Existen recursos para el español como Call Friend \cite{CALLFRIENDSpa}, Call Home \cite{CALLHOMESpa}, DIMEx100 \cite{Pineda2004DIMEx100:Spanish}, Heróico \cite{heroico}, Voxforfe \cite{Voxforge.org}, Fisher Spanish \cite{FischerSpa}, Albazín \cite{CampilloAlbayzinEvaluation} y CIEMPIESS \cite{Hernandez-MenaCIEMPIESS:Corpus}, los cuales se muestran en la tabla \ref{tab:spanish_corpora}, sin embargo estos recursos al compararse con los recursos de inglés se evidencia una diferencia en duración considerable.

\begin{table}[H]
\centering
\caption{Corpus hablados la lengua}
\label{tab:spanish_corpora}
\begin{tabular}{|l|l|l|l|l|}
\textbf{Base de datos} & \textbf{Duración} & \textbf{Vocabulario} & \textbf{Anotación} & Licencia\\
\hline
Fisher Spanish \cite{FischerSpa}  & 163 & Muy grande & Declaración  & LDC Non-Members\\
\hline
CALL FRIEND Spanish \cite{CALLFRIENDSpa}  & 85 & Grande & Declaración & LDC Non-Members\\
\hline
CALL HOME Spanish \cite{CALLHOMESpa}  & 70 & Grande & Declaración & LDC Non-Members\\
\hline
Voxforge \cite{Voxforge.org}  & 51 & Mediano & Declaración & GPL\\
\hline
CIEMPIESS\cite{Hernandez-MenaCIEMPIESS:Corpus}  & 17 & Grande & Declaración & CC BY-SA 4.0\\
\hline
Heroico \cite{HeroicoCorpus}  & 13 & Grande & Declaración & LDC Non-Members  \\
\hline
DIMEx100\cite{Pineda2004DIMEx100:Spanish}  & 5.6 & Medio & Fonético & \multicolumn{1}{|p{4cm}|}{Gratis para propósitos académicos}\\
\hline
Albayzín\cite{CampilloAlbayzinEvaluation}  & No especificado  & Grande & Declaración & ELRA\\
\hline
\end{tabular}
\end{table}

Se observa en ambas tablas que la mayoría de recursos son de licencia cerrada, sin embargo se destaca el corpus Libri Speech \cite{LIBRISPEECH}, el cual recopila mas de 1000 horas y está distribuido bajo la licencia Creative Commons versión 4. Este corpus se ha convertido en un recurso usado como línea base en múltiples investigaciones \cite{libribox_benchmark1,librilight,libribox_benchmark3}. 

La presencia de este corpus, validada por investigaciones posteriores abre la puerta a desarrollos similares para otros lenguajes utilizando la misma fuente de datos pública Libri Vox. 

% Aunque  existen notables diferencias entre los corpus del español y del inglés, el lenguaje español no se considera un lenguaje con escasos recursos \macb{\st{, pues e}. E}xisten múltiples corpus, bancos de árboles, datos anotados y transcritos, diccionarios y gramáticas formales \cite{CavarGlobalGORILLA}. 


\section{Objetivos}


\subsection{Objetivo general}

Implementar un alineador forzado con el fin de generar automáticamente un corpus para la lengua española de gran vocabulario y larga duración a partir de recursos existentes

\subsection{Objetivos específicos}

\begin{enumerate}
    \item Estudiar los algoritmos para alineadores forzados y sus implementaciones en el estado del arte de forma que soporten la implementación de uno para la lengua española.
    \item Recolectar recursos hablados existentes y de licencia abierta existentes para la lengua española.
    \item Construir un corpus de prueba para medir el desempeño de alineadores forzados en Español.
    \item Implementar un alineador forzado que anote a nivel de palabras un corpus de gran vocabulario y larga duración para la lengua española.
    \item Diseñar de manera empírica un conjunto de pruebas con el fin de establecer el nivel de anotación del alineador

\end{enumerate}

\section{Distribución del documento}

En el capítulo 2 se muestra una recolección de los recursos existentes y abiertos para el español, realizando un análisis detallado de cada uno de ellos.

En el capítulo 3 se diseñan los recursos de prueba para evaluar el rendimiento del alineador forzado propuesto.

El capítulo 4 muestra múltiples experimentos para los procesos de segmentación y alineación sobre las grabaciones de Libri Vox
\chapter{Revisión de recursos abiertos para el español}

La investigación en tecnologías del habla depende fuertemente en recursos de lenguaje de alta calidad y gran tamaño diseñados para la lengua específica. Los recursos abiertos de este tipo son más escasos en comparación con recursos de pago y la creación de dichos recursos para tareas como reconocimiento y síntesis de voz es costosa, por lo que muchos investigadores optan por crear conjuntos de datos pequeños y enfocados a su investigación.

Para lenguajes diferentes al inglés, la diferencia en disponibilidad de dichos recursos es aún mayor. Este capítulo presenta una compilación de recursos abiertos para tareas relacionados con tecnologías de voz para la lengua española, describiendo cada recurso y proponiendo posibles aplicaciones para cada corpus. Para seleccionar los recursos aquí incluídos se usaron como criterios la visibilidad, disponibilidad y el impacto.

% Veo que todo hace parte de la introducción del capítulo. MACB
% \section{Preámbulo}

Los modelos utilizados regularmente para procesamiento de voz requieren cantidades masivas de datos para lograr buenos resultados. En reconocimiento de voz, sistemas como Listen-Attend-Spell de Google \cite{Chan2016} fue entrenado utilizando 2000 horas aumentadas 20 veces con técnicas de aumento de datos hasta tener un corpus de 40000 horas. Otros modelos del tipo secuencia a secuencia usan 12500 horas de grabación \cite{Chiu2018}. Otros recursos reconocidos en la literatura como  TIMIT \cite{TIMIT} o Swichboard \cite{Switchboard} tienen una duración menor pero su anotación es a nivel fonético o de palabra.

Aunque los esfuerzos para reducir los requerimientos por grandes volúmenes de datos anotados usando aprendizaje semi supervisado \cite{AmazonSemiSupervised} y no supervisado \cite{ZeroResources}, estas aproximaciones requieren una gran cantidad de datos no supervisados.

Estas complicaciones se hacen mas evidentes en lenguas distintas al inglés donde los recursos son limitados en tamaño, variedad y nivel de anotación \cite{HernndezMena2017}. De igual manera, en comparación con el inglés que  tiene lineas base de investigación desde hace mas de dos décadas con corpus como TIMIT \cite{TIMIT}, Switchboard \cite{Switchboard} o Fisher \cite{Fisher}, para el español las líneas bases para investigación no están claramente definidas usualmente usan recursos multilingues y más limitados \cite{euronews_multilingual,librilight}.

La disponibilidad de los recursos para la investigación es un aspecto fundamental en el desarrollo de prototipos, algoritmos y técnicas. La tabla \ref{tab:resources_by_langauge} compara el n\'umero de recursos de licencias abiertas y de pago en los dos catálogos más comunes de recursos de lenguaje, el Linguistic Data Consortium (LDC) y la European Language Resources Association (ELRA), respectivamente. Podemos ver que el inglés tiene cinco veces más recursos que el español y el mandarín \cite{HernndezMena2017}.

\begin{table}[H]
\centering
\caption{Recursos para los cinco lenguajes más hablados del mundo \cite{HernndezMena2017}}
\label{tab:resources_by_langauge}
\begin{tabular}{|l|l|r|r|} % Cambié alineación para que los números sean comparables.
\hline
\textbf{Posicion} & \textbf{Lenguaje} & \textbf{LDC} & \textbf{ELRA} \\ \hline
1    & Mandarin & 24  & 6    \\ \hline
2    & Inglés  & 116 & 23   \\ \hline
3    & Español  & 20  & 20   \\ \hline
4    & Hindi    & 9   & 4    \\ \hline
5    & Árabe   & 10  & 32   \\ \hline
\end{tabular}
\end{table}

% Moví este párrafo arriba. MACB
%Este capítulo muestra una revisión de los recursos de licencia abierta para tareas de procesamiento de voz en español, usando como criterios la visibilidad, disponibilidad y el impacto, indicando que existen otros recursos no considerados por falta de los criterios mencionados previamente.


\section{Recursos abiertos para el español}

Esta sección menciona los recursos de lenguaje abiertos para la lengua española con licencias abiertas y distribuidos abiertamente en la web. Un resumen se presenta en la tabla \ref{tab:open_source_spanish_corpus} organizada por fecha de publicación; mostrando características importantes de cada conjunto de datos, como la licencia de distribución, el nivel de anotación, la duración, las condiciones de grabación, el dialecto y la fuente de las grabaciones.

En la tabla \ref{tab:open_source_spanish_corpus} se utilizan las siguientes abreviaciones:  CC BY-SA 4.0 Creative Commons Share A Like v4.0 license, LDC Non-Members LDC User LDC User Agreement for Non-Members, para dialectos: Mexicano (MX), Peninsular (ES), Argentino (AR), Chileno (CH), Peruano (PE), Colombiano (CO), Puerto Riqueño (PR), Venezolano (VE). 


\begin{table*}[ht]
\caption{Lista de recursos abiertos para el Español}
\label{tab:open_source_spanish_corpus}
\begin{tabular}{|l|l|l|l|l|l|l|l|}
\hline
\textbf{Nombre} & \textbf{Licencia}  & \multicolumn{1}{|p{1.8cm}|}{\textbf{Anotación}} & \textbf{Duración} & \multicolumn{1}{|p{2cm}|}{\textbf{Condiciones de Grabación}} & \textbf{Dialectos} & \textbf{Fuente} &  \multicolumn{1}{|p{2cm}|}{\textbf{Publicación}} \\ \hline

{DIMEx100}  & \multicolumn{1}{|p{2cm}|}
            {Gratis para propósitos académicos}        & {Fonético} & {5h} & {Estudio} & {MX}                       & {Internet} & 2004 \\ \hline
{Heroico}  & \multicolumn{1}{|p{2cm}|}
{LDC Non-Members}              & {Declaración} & {13h}  & {Ruidoso}  & {MX}                  & \multicolumn{1}{|p{2cm}|}
                                                                                               {Libros} &                                                         2006\\ \hline 

{CIEMPIESS}   & \multicolumn{1}{|p{2cm}|}
             {CC BY-SA 4.0}            & {Declaraión}     & {17h}  & {Estudio} & {MX}                 & {Radio} & 2014 \\ \hline
\multicolumn{1}{|p{2cm}|}
{CIEMPIESS Light}   & \multicolumn{1}{|p{1.5cm}|}
             {CC BY-SA 4.0}            & {Declaraión}     & {18h}  & {Estudio} & {MX}                 & {Radio} & 2016 \\ \hline
\multicolumn{1}{|p{2cm}|}
{CIEMPIESS Balance}   & \multicolumn{1}{|p{1.5cm}|}
             {CC BY-SA 4.0}            & {Declaraión}     & {18h}  & {Estudio} & {MX}                 & {Radio} & 2018 \\ \hline
\multicolumn{1}{|p{2cm}|}
{CIEMPIESS EXP. - Complementary}   & \multicolumn{1}{|p{1.5cm}|}
             {CC BY-SA 4.0}            & {Declaraión}     & {1h}  & {Estudio} & {MX}                 & {Radio} & 2018 \\ \hline
\multicolumn{1}{|p{2cm}|}
{CIEMPIESS EXP. - FEM}   & \multicolumn{1}{|p{1.5cm}|}
             {CC BY-SA 4.0}            & {Declaraión}     & {13h}  & {Estudio} & {MX}                 & {Radio} & 2018 \\ \hline
\multicolumn{1}{|p{2cm}|}
{CIEMPIESS EXP. - Test}   & \multicolumn{1}{|p{1.5cm}|}
             {CC BY-SA 4.0}            & {Declaraión}     & {8h}  & {Estudio} & {MX}                 & {Radio} & 2018 \\ \hline

{M-AILABS}  & \multicolumn{1}{|p{1.5cm}|}
            {M-AILABS BSD License}               & {Declaración} & {108h} & {Ruidoso}  & \multicolumn{1}{|p{1.5cm}|}
            {ES, MX, AR, Otros} 
                                                                                                                & {LibriVox} & 2019 \\ \hline
\multicolumn{1}{|p{2cm}|}
{Google Language 
Resources Latin American} & \multicolumn{1}{|p{1.5cm}|}
                        {CC BY-SA 4.0}  & {Declaración} & {37h}  & {Estudio} & \multicolumn{1}{|p{1.5cm}|}
            {AR, CH, CO, PE, PR, VE}       
                                                                                                                &{Internet}   & 2020\\ \hline
                                                                                                    


{Common Voice}  & {CC-0}                            & {Declaración} & {529h} & {Ruidoso}  &  \multicolumn{1}{|p{1.5cm}|}
            {ES, MX, AR, Others}                                                          & {Internet} & 2020 \\ \hline                

{Librivox Spanish}  & \multicolumn{1}{|p{1.5cm}|}
{LDC Non-Members}               & {Declaración} & {73h} & {Ruidoso}  &  \multicolumn{1}{|p{1.5cm}|}
            {ES, MX, AR, Others}                                                          & {Internet} & 2020 \\   \hline     
\end{tabular}
\end{table*}

\subsection{DIMEx100}

DIMEx100 \cite{Pineda2004DIMEx100:Spanish} es un corpus oral diseñado y grabado por el Instituto de Investigaciones en Matemáticas Aplicadas y en Sistemas IIMAS de la Universidad Nacional Autónoma de México UNAM en 2004. DIMEx100 utiliza textos del Corpus 230 \cite{Corpus230}, un corpus escrito fonéticamente equilibrado para el español. Este proyecto fue anotado usando un nuevo diccionario fonético para el español mexicano llamado MEXBET, que mejora la representación de los alófonos del español mexicano en comparación con otros alfabetos fonéticos multilingües como SAMPA y WordNET \cite{mexbet}. El proceso de grabación para DIMEx100 se realizó en un estudio de grabación, con bajo nivel de ruido, con el micrófono ubicado a una distancia homogénea de los altavoces. Se seleccionaron 100 hablantes entre 16 y 36 años siendo 50 hombres y 50 mujeres. La edad media de los hablantes fue de 23 años y todos los hablantes hablan el dialecto mexicano.

Este es un corpus con 5.6 horas de duración, con equilibrio fonético y de género y conformado por 5010 declaraciones con anotaciones fonéticas. Este corpus se puede usar para el reconocimiento de palabras aisladas o como recurso inicial para la construcción corpus más grandes. Sin embargo, todos los locutores de este corpus usan el dialecto mexicano, lo que podría generar inconvenientes al realizar tareas donde los locutores usen otro dialecto.

\subsection{Heroico}

Heroico \cite{heroico} es un corpus hablado publicado en 2006 por John Morgan en el Departamento de Lenguas Extranjeras (DFL) y el Centro para el Aprendizaje de Idiomas Mejorado por la Tecnología (CTELL). Todas las grabaciones se llevaron acabo en El Heroico Colegio Militar y la Academia Militar Mexicana en México. Este corpus tiene un desequilibrio de género y se anota a nivel de declaración. El corpus de HEROICO está compuesto por 102 hablantes que responden abiertamente a un conjunto de 143 preguntas y leen 75 expresiones de un conjunto de datos compuesto por 205 frases cortas y 519 oraciones simples de notas académicas de la escuela secundaria.

La distribución del corpus Heroico también incluye el subcorpus USMA, registrado en 1997 que está compuesto por 206 expresiones fijas leídas por 18 hablantes nativos y no nativos diferentes. La duración total del corpus es de 13 horas. Todas las grabaciones se realizaron con computadoras de escritorio, el ruido está presente en varias grabaciones.

Los corpus Heroico y USMA combinados tienen 13 horas de audio, y parte del corpus es habla espontánea en entornos ruidosos y diferentes potencias de señal, lo que permite que este corpus entrene modelos para un reconocimiento de voz robusto. Además, algunos hablantes en el corpus de la USMA no son hablantes nativos de español y hay diferentes dialectos de español que también dan variabilidad para los modelos independientes del hablante.

\subsection{Proyecto CIEMPIESS}

El Corpus de Investigación en Español de México del Posgrado de Ingeniería Eléctrica y Servicio Social de la Universidad Nacional Autónoma de México CIEMPIES-UNAM es un proyecto iniciado en 2012 con el objetivo de desarrollar y compartir herramientas gratuitas y de código abierto para el procesamiento del habla en el idioma español. El proyecto no se limita a la creación de corpus sino a la investigación en tecnologías de habla aplicadas al idioma español \cite{CIEMPIESS-Webpage}. Para la anotación de corpus, el proyecto CIEMPIESS utiliza estudiantes de pregrado para realizar anotaciones manuales. Toda la segmentación y anotación se realiza utilizando herramientas de código abierto. Adicionalmente, el corpus final es distribuido bajo la licencia abierta Creative Commons Share A Like versión 4. Este proyecto ha publicado varios corpus a lo largo de los años, incluido el corpus original de CIEMPIESS \cite {CIEMPIESS}, CIEMPIESS Light \cite {CIEMPIESS-LIGHT}, CIEMPIESS Balance \cite {CIEMPIESS-BALANCE}, CIEMPIESS Experimentation \cite {CIEMPIESS-Experimentation} y LibriVoxSpanish \cite {LibriVox-Spanish} que se describen a continuación.

\subsubsection{CIEMPIESS}

El Corpus de Investigación en Español de México del Posgrado de Ingeniería Eléctrica y Servicio Social CIEMPIESS \cite{CIEMPIESS} fue publicado en 2014 por el Departamento de Procesamiento Digital de Señales de la Universidad Nacional Autónoma de México, con el objetivo de crear un corpus hablado para el reconocimiento automático de voz. Todas las grabaciones fueron extraídas de diversos programas de radio emitidos por la Universidad, en total se seleccionaron 43 programas de una hora, los cuales fueron segmentados manualmente y sólo los segmentos de un solo locutor fueron seleccionados; segmentos con sonidos de fondo como música o interferencias fueron removidos. Dando la naturaleza del programa de radio, este corpus contiene discurso continuo y es desequilibrado de género, con 77.86\% grabaciones masculinas y solo 22.14\% grabaciones femeninas. El corpus completo consta de 16.717 grabaciones y tiene una duración total de 17 horas. El proceso de anotación se realizó usando el alfabeto fonético MEXBET vocales tónicas \cite{mexbet}. La anotación se realizó a nivel de palabras con formato TextGrid \cite{TextGrids}.

\subsubsection{CIEMPIESS Light}

CIEMPIES Light \cite{CIEMPIESS-LIGHT} se publicó después de dos años de experimentación con el CIEMPIESS Corpus. El equipo de CIEMPIESS recibió muchos comentarios y se lanzó una nueva versión del corpus original. En esta revisión se cambió el formato de distribución de archivos SPH a archivos WAV, también se eliminó una pequeña porción de grabaciones reportadas como problemáticas y se añadieron otras para compensar las eliminadas, añadiendo en total una hora m\'as de audio. Esta nueva distribución es compatible con herramientas ASR modernas como Kaldi y CMU Sphinx. La mayoría de las grabaciones originales se conservaron, pero otras se reemplazaron. 

Este corpus puede considerarse como la versión dos del corpus original de CIEMPIESS, pero mejorado para trabajar con sistemas ASR modernos El corpus está organizado por género y locutor (53 hombres y 34 mujeres en total) y anotado a nivel de declaración. 

\subsubsection{CIEMPIESS Balance}

CIEMPIESS Balance \cite{CIEMPIESS-BALANCE} fue creado considerando que el corpus de CIEMPIESS tiene un desequilibrio de género, se creó un corpus de nueva corpus a partir de grabaciones del mismo programa de radio para equilibrar el corpus de CIEMPIESS LIGHT. Este corpus está compuesto por 18 horas y 20 minutos donde 12 horas y 40 segundos son de hablantes mujeres y 5 horas y 40 minutos son de hablantes masculinos, para un total de 36 horas aproximadamente por cada género.


\subsubsection{CIEMPIES EXPERIMENTATION}

CIEMPIES Experimentación \cite{CIEMPIESS-Experimentation} es un conjunto de tres corpus distribuidos en uno solo, con un corpus para alófonos de equilibrio fonético para el español mexicano, un corpus con s\'olo mujeres hablantes y uno diseñado para ser un corpus de prueba estándar.

CIEMPIESS EXPERIMENTATION - COMPLEMENTARY es un corpus fonéticamente equilibrado para palabras aisladas que contiene una hora de audio anotado usando MEXBET 29 y MEXBET 66, dos anotaciones fonéticas que consideran diferentes alófonos para el idioma español. Este corpus contiene grabaciones de 10 locutores masculinos y 10 locutores femeninos y fue creado para mejorar los motores de reconocimiento de voz que no encontraron ocurrencias de alófonos específicos al entrenar modelos acústicos.

CIEMPIES EXPERIMENTATION  - FEM es un corpus con sólo locutoras femeninas, que contiene 13 horas y 54 minutos de grabaciones Hay un total de 21 locutoras diferentes: 16 hablantes de México, y 5 hablantes de otros dialectos (venezolano, argentino, salvadoreño, dominicano, y desconocido)

CIEMPIES EXPERIMENTATION  - TEST es un corpus con equilibrio de género diseñado para probar aplicaciones de habla. Cuenta con un total de 8 horas y 8 minutos de grabaciones de 10 locutoras mujeres y 10 locutores masculinos.

\subsubsection {Librivox Spanish}

Librivox Spanish \cite{LibriVox-Spanish} es un corpus en español basado en grabaciones abiertas subidas al sitio Librivox \footnote{http://librivox.org}, todos los audiolibros son de dominio público y parte del Proyecto Guttenberg \footnote{https://gutenberg.org} o liberados al dominio público. El corpus fue anotado manualmente por estudiantes de pregrado como parte de su requerimiento de servicio social en la Universidad Nacional Autónoma de México. Tiene un total de 73 horas de audio, balanceado por género con 60 horas de hablantes nativos de español y el resto de hablantes no nativos. Debido a su naturaleza de creación colectiva, las grabaciones tienen una calidad diferente; en la mayoría de los casos, las grabaciones se realizaron en un entorno silencioso y utilizando micrófonos de computadora normales.

Como este corpus se anota manualmente y las grabaciones de audio originales son de dominio público, el corpus en sí se puede utilizar como un corpus de prueba para la segmentación automática de las grabaciones originales. Además, el corpus tiene hablantes con diferentes dialectos, lo que hace que sea apropiado crear sistemas de reconocimiento de voz independientes del locutor.

\subsection{M-AILABS}

M-AILABS Speech Dataset \cite{M-AILABS} es un corpus hablado creado por Imdat Solak utilizando recursos disponibles abiertamente de LibriVox y Project Guttenberg. El corpus era multilingüe, incluido alemán, inglés estadounidense e inglés británico, español, italiano, ruso ucraniano, francés y polaco.

El subcorpus en español tiene 108 horas de duración y está  dividido en tres subcorpus: un corpus femenino de 10 horas, grabado por una locutora mexicana, un corpus masculino de 72 horas, grabado por un argentino y un español, y un corpus mixto de 25 horas, grabado por varios hablantes no identificados. Todas las grabaciones incluidas en este corpus también pertenecen al proyecto LibriVox.

Para la segmentación se utilizó una segmentación basada en silencios, determinando los valores mínimos de toda la grabación en decibeles y cortando en segmentos de intensidad menor. Posteriormente utilizando un servicio de reconocimiento automático de voz en cada segmento se evalúan manualmente los resultados obtenidos contra el texto original, dejando al final una transcripción corregida por un humano.



\subsection{Google TTS Latin American}

El corpus de Crowdsourcing Latin American Spanish for Low-Resource Text-to-Speech \cite{googleTTSLatinAmericanSpanishCorpus} fue creado por Google Research y el Laboratoire de Sciences Cognitives et Psycholinguistique y Graduate School of Engineering de la Universidad de Tokio. El corpus se diseñó para tener varios dialectos con alta calidad para los sistemas de texto a voz de América Latina. Este corpus incluye 6 subcorpus para dialectos argentino, chileno, colombiano, peruano, puertorriqueño y venezolano, y un total de 174 hablantes y 37,7 horas de audio. Las oraciones se seleccionaron con base en un sistema de conversación para el español mexicano. Posteriormente las oraciones se adaptaron a cada dialecto en particular y solo 30 se mantuvieron como canónicas las cuales todos los locutores debían grabar.

Las grabaciones se realizaron en una cabina vocal portátil con un micrófono de condensador en un entorno cercano al silencio.

\subsection{Common Voice}

La fundación de software Mozilla creó a mediados del 2017 una plataforma de crowdsourcing para recopilar recursos de voz. Estos recursos hacen parte del proyecto Deep Speech Project, basado en la propuesta de Baidu Deep Speech \cite{deepspeeh}. Posterioromente la iniciativa se consolidó como un proyecto independiente llamado Common Voice \cite{Common-Voice}.

Usando una plataforma web \footnote{https://commonvoice.mozilla.org/}, los usuarios graban frases cortas y validan las grabaciones de otros usuarios. La última versión  incluyó 60 idiomas, incluidos entre ellos el inglés, alemán, francés y chino. El idioma español cuenta con 579 horas de audio.

Considerando que el corpus es grabado y validado por voluntarios, las condiciones de grabación no son homogéneas y pueden contener errores de anotación o ruido de fondo.

\subsection{Vox Populli}

Vox Populli \cite{voxpopulli} es un corpus publicado por Facebook anotando automáticamente las grabaciones y traducciones realizadas en el parlamento europeo. El corpus está separado en 16 lenguajes, con 160 horas de grabaciones anotadas a nivel de declaración de 305 locutores distintos y 4400 horas no anotadas para el idioma español. 

El procesamiento de las horas anotadas consistió en el uso de las transcripciones manuales de las sesiones. Como estas transcripciones tenían problemas de alineación, sobre cada audio, que corresponde a un párrafo se realizó una segmentación por silencios de duración máxima de 20 segundos. La duración aproximada de cada párrafo original es de 197 segundos. Los nuevos subsegmentos fueron anotados automáticamente con un software de reconocimiento automático de voz. Se comparó el resultado obtenido por el software reconocimiento automático de voz con la anotación esperada en la transcripción manual y los resultados individuales con precisión superior al 80\% fueron conservados.



\section{Herramientas existentes para la anotación automática}
% \section{Existing tools}

Muchos Alineadores Forzados de código abierto utilizan investigación teórica para materializar la alineación de recursos existentes. Basada en la categorización mencionada previamente, se agrupan alineadores existentes en las siguientes categorías

\begin{itemize}
    \item Pliegues Dinámicos Temporales
    \item Modelos Ocultos de Markov
    \item Redes Neuronales Artificiales
\end{itemize}
% Open source forced aligners uses theoretical research to create tools to materialize alignment on existing resources. Based on previously mentioned categorization the existing aligners are grouped in the categories: Dynamic Time Warping, Hidden Markov Models and Artifical Neural Networks.

\subsection{Pliegues Dinámicos Temporales}
% \subsection{Dynamic Time Warping}

Para DTW la idea principal es generar señales de voz artificiales utilizando software de tipo grafema a fonema, también conocido como texto a vos (Text To Speech TTS), y luego aliear la señal de entrada con la generada artificialmente.

% For DTW the main idea is to generate an artificial speech using Grapheme to Phoneme  software and then align the input speech with the artificially generated wave

\textbf{Aeneas}.

Aneneas \cite{aeneas} es un software desarrollado por Alberto Pettarin para Read Beyond, un software de audio libros, liberado bajo licencia GNU Affero General Public Licence versión 3 (AGPL v3) en 2015, con varias actualizaciones hasta 2017. Para la generación de la señal artificial, Aeneas usa eSpeak \cite{espeak}, otro software licenciado bajo la licencia GNU Public Licence (GPL) por the Free Software Foundation. eSpeak en su versión original soporta más de 28 lenguajes, incluyendo Inglés y Español
% Aeneas \cite{aeneas} uses espeak \cite{espeak} to generate the base generates speech wave. Then align the input word using dynamic programming.

\subsection{Modelos Ocultos de Markov}
% \subsection{Hidden Markov Models}

Los Modelos Ocultos de Markov usan una serie de algoritmos para parametrizar los modelos, entre los cuales se destacan el algoritmo de Viterbi para la decodificación \cite{Forney1973TheAlgorithm} y el algoritmo Baum-Welch para la estimación inicial de parámetros.

Este conjunto de algoritmos son implementados por paquetes especializados para el reconocimiento automático del habla, como  HTK \cite{Young1994ThePhilosophy}, Julius \cite{LeeEurospeechEngine} y CMU Sphinx \cite{Lee1990AnSystem}. Estas implementaciones abiertas de algoritmos son usadas por algunas herramientas mostradas a continuación.
% Work with HMM uses a basic set of algorithms where the Viterbi algorithm \cite{Forney1973TheAlgorithm} and the Baum Welch Algorithm. These set of algorithms are implemented for ASR in frameworks and toolkits like HTK \cite{Young1994ThePhilosophy}, Julius \cite{LeeEurospeechEngine} and CMU Sphinx \cite{Lee1990AnSystem}.

\textbf{MAUS}
El Segmentador Automático de Munich (Munich AUtomatic Segmentation) desarrollado por el Instituto de Fonética y procesamiento de señales de la Universidad de Munich está construido usando herramientas de decodificación de HTK y BALLON un sistema de grafemas a fonemas también desarrollado por la  universidad de Munich. Este paquete fue desarrollado en 1994 por Uwe Reichel y Florian Schiel licenciado para propósitos no comerciales y académicos. \cite{WesenickAPPLYINGPRONUNCIATION}. 
% The Munich Automatic Segmentation \cite{WesenickAPPLYINGPRONUNCIATION} is a build on top of HTK and BALLON a Grapheme to Phoneme Suite created by Uwe Reichel and uses a hybrid approach between DTW and HMM

\textbf{SPPAS}
SPPAS es un anotador automático y analizador de habla es una herramienta de computación científica desarrollada para proveer un análisis fonético robusto y confiable \cite{Bigi2016ASPPAS}. Fue creada por Brigitte Bigi  en el Laboratorio del habla y la lengua (Laboratoire Parole et Langage) de Francia, licenciado con la licencia GPL v3, utiliza Julius para el procesamiento de los Modelos Ocultos de Markov.
% SPPAS \cite{Bigi2016ASPPAS} is a suite for automated for automated annotation and speech analysis created by Brigitte Bigi at Laboratoire Parole et Langage in France and uses Julius for HMM processing and Viterbi Algorithm

\textbf{Prosodylab Aligner}
Prosodylab Aligner \cite{Gorman2011Prosodylab-aligner:Speech} fue creado en el Laboratorio de Prosodia (Prosody Lab) en la Universidad de Mcgill en Canada, por Kyle Gorman y está compuesta de una serie de herramientas y scripts para crear alineaciones usando HTK usando monófonos para entrenar sus modelos.
% Prosodylab Aligner \cite{Gorman2011Prosodylab-aligner:Speech} was created at Prosody Lab in Mcgill University by Kyle Gorman and are composed by a set of tools of scripts to create alignment using HTK as backend using mono-phones to train its models

\subsection{Redes Neuronales Artificiales}
% \subsection{Artificial Neural Networks}

Los anotadores de código abierto basados en redes neuronales utilizan Kaldi \cite{Povey_ASRU2011}, un paquete diseñado por Daniel Povey en el instituto universitario John Hopkings, y licenciado bajo la licencia Apache 2.
% For ANN the Kaldi toolkit \cite{Povey_ASRU2011} is used to improve development times.   

\textbf{Gentle Forced Aligner}

Gentle \cite{gentle} es un alineador forzado construido a partir de un modelo de bigramas y conceptos de programación dinámica para alinear el audio.
% Gentle \cite{gentle} is a FA build on top of Kaldi. The main approach is to use a web server to evaluate an input speech. The model uses dynamic programming to align audio using a bigram model

\textbf{The Montreal Forced Aligner}

El Alineador Forzado de Montreal, fue creado en la Universidad de Mcgill en el laboratorio de prosidia \cite{McAuliffe2017MontrealKaldi}, licenciado con la licencia MIT  como una versión mejorada de Prosylab Aligner, utilizando el corpus GlobalPhone  para crear modelos trifónicos que mejoran el rendimiento con respecto al alineador anterior.
% The Montreal Forced Aligner \cite{McAuliffe2017MontrealKaldi} was created at Prosody Lab in Mcgill University as the evolution for Prosodylab Aligner. It uses Kaldi as backend and Globalphone to create a triphone model that improves performance comparing Prosodylab Aligner
\chapter{Creación de recursos de prueba}

Con el objetivo de determinar una línea base para la creación del corpus, se inicia realizando una segmentación de recursos abiertos existentes para medir posteriormente la calidad de las segmentaciones automáticas.

Se define segmentación del audio como el proceso por el cual, a partir de una grabación de voz, se particiona la misma en segmentos mas pequeños, siendo particiones a nivel de declaración, palabra o fonético. Las anotación es el proceso por el cual a partir de una grabación de voz se relaciona el contenido linguístico de la misma

Se realizaron dos anotaciones para el desarrollo de la investigación: una anotación a nivel fonético y otra a nivel de declaración.

Para ambas anotaciones se utilizó el software Praat \cite{Praat} desarrollado por Paul Boersma y David Weenink de la Universidad de Ámsterdam. Con el cual de manera visual es posible crear archivos de anotación del audio. Estos archivos usan el formato TextGrid, especificado por la misma herramienta, donde se definen secuencias de elementos que identifican el inicio, finalización y texto encontrado en cada segmento. Estos archivos son almacenados en formato de texto para su fácil lectura \cite{TextGrids}

\section{Anotación manual a nivel fonético}

Se define fonema como la unidad fonológica básica, que describe una interacción específica del aparato fonador. De esta manera las posibles combinaciones finitas que puede ejecutar el ser humano para emitir sonidos de voz quedan acotadas por estas representaciones, y a su vez son clasificadas en función de la cercanía de las interacciones.

Al ser el lenguaje hablado un subconjunto de las señales auditivas percibas por el oído como variación en la presión del aire, los primeros fonemas que destacan son aquellos que modifican directamente esta presión del aire. El aparato fonador utiliza los pulmones y el diafragma para expulsar aire, luego utiliza las cuerdas vocales, y la boca para modificar esta presión generando variaciones en los sonidos percibidos. Estos sonidos primarios son llamados vocales y se pueden caracterizar por la apertura de la boca y el lugar de resonancia de la señal. Se puede observar un diagrama del aparato fonador en la figura \ref{img:aparato_fonador} tomada de la conferencia Hablemos de Voz \cite{hableomsDeVoz}.

\begin{figure}[H]
\caption{Aparato Fonador.}
\label{img:aparato_fonador}
\includegraphics[width=\textwidth]{imagenes/03_02_aparato_fonador.png}
\end{figure}

El Alfabeto Fonético Internacional o IPA por sus siglas en inglés, fue estructurado desde 1888 y representa todas las posibles configuraciones para los fonemas que pueden ser ejecutados por el ser humano y la caracterización e identificación que da a las vocales es la que se puede observar en la tabla \ref{tab:ipa_table_vowels}


% \begin{landscape}
\begin{table}
\centering
\caption{Alfabeto Fonético Internacional: Vocales}
\label{tab:ipa_table_vowels}
\begin{tabular}{|l|l|l|l|l|l|l|}
\hline
{} & \multicolumn{2}{|c|}{Frontal} & \multicolumn{2}{|c|}{Central} & \multicolumn{2}{|c|}{Posterior}   \\
\hline
Cerrada & i  & y &\textbaru   & \textbari & \textturnm &  u  \\
\hline
Casi cerrada & \textsci  & \textscy &  \multicolumn{2}{|c|}{} &  & \textupsilon  \\
\hline
Semicerrada & e  & \textipa{\o} &\textreve &  \textbaro & \textramshorns & o \\
\hline
Intermedia &  \multicolumn{2}{|c|}{} & \multicolumn{2}{|c|}{\textschwa} &  \multicolumn{2}{|c|}{} \\
\hline
Semiabierta &\textepsilon  & \textipa{\oe} & \textrevepsilon & \textcloserevepsilon  & \textturnv & \textopeno \\
\hline
Casi abierta & \multicolumn{2}{|c|}{\ae} &\multicolumn{2}{|c|}{\textturna} &  \multicolumn{2}{|c|}{}  \\
\hline
Abierta & a  & \textscoelig &  \multicolumn{2}{|c|}{}  &\textscripta & \textturnscripta \\
\hline
\end{tabular}
\end{table}
% \end{landscape}

Además de estas alteraciones principales en las perturbaciones del aire, existen otros sonidos que interrumpen el flujo de aire emitido por el diafragma. Estos sonidos son denominados consonantes y son clasificados por su lugar de articulación y el tipo de articulación. El IPA también define una clasificación para las consonantes, la cual puede observarse en la tabla \ref{tab:ipa_table_pulmonic_consonants} y la tabla \ref{tab:ipa_table_non_pulmonic_consonants}

\begin{landscape}
\begin{table}
\centering
\caption{Alfabeto Fonético Internacional: Consonantes pulmónicas}
\label{tab:ipa_table_pulmonic_consonants}
\begin{tabular}{|p{25mm}|l|p{15mm}|l|l|p{15mm}|l|l|l|l|l|l|}
\hline
{} & Bilabial & Labio\newline dental & Dental & Alveolar & Post-\newline alveolar & Retrofleja & Palatal & Velar & Uvular & Faríngea & Glotal \\
\hline
Plosiva& p b  & & \multicolumn{3}{|c|}{t d} & \textipa{\:t \:d } & \textipa{c \*j} & k g &  q G & & \textipa{P} \\
\hline
Nasal& m &  \textipa{M} & \multicolumn{3}{|c|}{n} & \textipa{\:n}  &  \textipa{\*n}  & \textipa{N} & N &  & \\
\hline
Vibrante& B & & \multicolumn{3}{|c|}{r}  & & & & R &  & \\
\hline
Aproximante & & \textipa{ⱱ} & \multicolumn{3}{|c|}{\textipa{R}} & \textipa{\:r} & & & & &  \\
\hline
Fricativa  & \textipa{F B}& f v & \textipa{T D} & s z & \textipa{S z} & \textipa{\:s \:z} & \textipa{\c{c} J}& x \textipa{G} &\textipa{X  K}  &\textcrh \textipa{Q} & h\textipa{H}  \\
\hline
Lateral \newline fricative& & & \multicolumn{3}{|c|}{\textbeltl \textipa{\*z}} & & & & &  & \\
\hline
Aproximante & & \textipa{V}& \multicolumn{3}{|c|}{\textipa{\!R}} & \textipa{\:R} & j  & \textturnmrleg & & &  \\
\hline
Aproximante lateral& & \multicolumn{3}{|c|}{\textipa{l}} &  \textraisevibyi & \textturny & \textipa{\;L} & & & & \\
\hline
\end{tabular}
\end{table}
\end{landscape}

% \begin{landscape}
\begin{table}
\centering
\caption{Alfabeto Fonético Internacional: Consonantes no pulmónicas}
\label{tab:ipa_table_non_pulmonic_consonants}
\begin{tabular}{|p{20mm}|l|l|l|l|l|l|}
\hline
{} & Bilabial & Dental & Alveolar  & Palatal & Velar & Uvular   \\
\hline
Eyectiva \newline oclusiva& p\textipa{'}  & &t   & c\textipa{'} & k\textipa{'} &  q\textipa{'}  \\
\hline
Eyectiva \newline fricativa& \textipa{F'}  & \textipa{T'}&  s\textipa{'} & \textipa{\c{c}'} & x\textipa{'} & \textipa{X'}  \\
\hline
Click &\textipa{\!o}  & \textipa{|} & \textipa{!} &  & & \\
\hline
Implosiva & \textipa{\!b}  & \multicolumn{2}{|c|}{\textipa{\!d}} &  \textipa{\!j} & \textipa{\!g} & \textipa{\!G} \\
\hline
\end{tabular}
\end{table}
% \end{landscape}

El español utiliza solamente ciertos fonemas del IPA, y la representación de este subconjunto se define como alfabeto fonético para el español. Dado que el IPA contiene símbolos por fuera de unicode, el tratamiento computacional de los alfabetos fonéticos genera problemas; por lo cual existen adecuaciones del mismo a unicode, como es el caso de SAMPA \cite{SAMPA}. 

Otros diccionarios fonéticos para el español han sido propuestos, como es el caso de MEXBET \cite{mexbet}, un alfabeto fonético para el español representable en unicode. En la tabla \ref{tab:mexbet} se muestra MEXBET, segmentado por el tipo de articulación y el modo de articulación, relacionandolo con el símbolo correspondiente del IPA en paréntesis

\begin{table}[H]
\centering
\caption{Mexbet}
\label{tab:mexbet}
\begin{tabular}{|l|l|l|l|l|l|l|}
\textbf{Consonantes}         & \textbf{Labiales}     & \textbf{Labiodentales} & \textbf{Dentales}     & \textbf{Alveolares}    & \textbf{Palatales}    & \textbf{Alveolares}  \\ \hline
Oclusivos Sordos             & p    &       & t    &       &      & k   \\ \hline
Oclusivos Sonoros            & b    &       & d    &       &      & g   \\ \hline
Africado Sordo               &      &       &      &       & tS   &     \\ \hline
Fricativos Sordos            &      & f     &      & s     &      & x   \\ \hline
Fricativos Sonoros           &      &       &      &       & Z    &     \\ \hline
Nasales     & m    &       &      & n     & n$\sim$               &     \\ \hline
Lateral &  &  &  & l & & \\ \hline
\end{tabular}
\end{table}

Las grabaciones seleccionadas para el corpus fonético de prueba se extrajeron del Open Speech Corpus \cite{Collazos2015} y el subcorpus de palabras aisladas, el cual está compuesto por 334 palabras distintas grabadas por múltiples locutores con un total de 9441 grabaciones de 39 locutores distintos, seleccionando aleatoriamente 100 palabras distintas y realizando una anotación manual por medio de PRAAT \cite{Praat}.

La anotación se realizó utilizando la convención definida en la tabla \ref{tab:anotacion_fonetica}. Esta convención se us\'o considerando que las grabaciones seleccionadas fueron grabadas en su totalidad por locutores latino americanos, del país Colombia y la región del Valle del Cauca. 

\begin{table}[H]
\centering
\caption{Anotación Fonética}
\label{tab:anotacion_fonetica}
\begin{tabular}{|l|l|l|}
\hline
\textbf{Simbolo} & \textbf{Letra} & \textbf{Representación} \\ \hline
sil              & & Silencio                               \\ \hline
a                & a & vocal abierta central                   \\ \hline
b                & b &  consonante plosiva bilabial sonora   \\ \hline 
k                & c &  consonante plosiva palatal no sonora \\ \hline 
S                & ch &  consonante fricativa palatal        \\ \hline 
d                & d &  consonante plosiva dental sonora     \\ \hline 
e                & e &  vocal semi-abierta central              \\ \hline
f                & f &  consonante fricativa labiodental     \\ \hline 
g                & g &  consonante plosiva velar sonora      \\ \hline
i                & i &  vocal cerrada frontal                   \\ \hline
j                & j &  consonante approximante palatal       \\ \hline
l                & l &  consonante approximante alveolar      \\ \hline 
m                & m &  consonante nasal bilabial            \\ \hline 
n                & n &  consonante nasal alveolar            \\ \hline 
N                & ñ &  consonante nasal palatal             \\ \hline 
o                & o &  vocal semi-cerrada posterior               \\ \hline
p                & p &  consonante plosiva no sonora         \\ \hline 
R                & r &  consonante vibrante alveolar sonora   \\ \hline 
r                & r &  consonante vibrante alveolar no sonora \\ \hline
s                & s &  consonante fricativa alveolar         \\ \hline
t                & t &  consonante plosiva dental no sonora   \\ \hline
u                & u &  vocal cerrada posterior                    \\ \hline
y                & y &  consonante fricativa palatal         \\ \hline
\end{tabular}
\end{table}

La distribución de fonemas en la anotación manual se muestra en la tabla \ref{tab:distribucion_fonetica}

\begin{table}[H]
\centering
\caption{Distribución fonética}
\label{tab:distribucion_fonetica}
\begin{tabular}{|l|l|}
sil & 173 \\ \hline
a   & 91  \\ \hline
o   & 79  \\ \hline
e   & 59  \\ \hline
n   & 45  \\ \hline
i   & 40  \\ \hline
l   & 31  \\ \hline
s   & 29  \\ \hline
t   & 27  \\ \hline
d   & 22  \\ \hline
R   & 22  \\ \hline
b   & 20  \\ \hline
r   & 19  \\ \hline
p   & 19  \\ \hline
j   & 16  \\ \hline
u   & 16  \\ \hline
m   & 15  \\ \hline
k   & 14  \\ \hline
g   & 11  \\ \hline
f   & 7   \\ \hline
c   & 5   \\ \hline
y   & 5   \\ \hline
C   & 2   \\ \hline
N   & 1   \\ \hline
S   & 1   \\ \hline
\end{tabular}
\end{table}

Las grabaciones seleccionadas se distribuyen entre 22 grabadas por locutores de género femenino, 68 masculinos y 10 no identificado.

En la imagen \ref{img:anotacion_fonetica_praat} se muestra la anotación fonética de una grabación que contiene la palabra arete realizada en Praat y en el archivo \ref{file:text_grid} el archivo correspondiente a la anotación fonética. En adelante se denominará a esta anotación corpus fonético de experimentación.


\begin{figure}[H]
\caption{Anotación fonética con Praat}
\label{img:anotacion_fonetica_praat}
\includegraphics[width=\textwidth]{imagenes/03_01_anotacion_fonetica.png}
\end{figure}

\lstinputlisting[caption={Archivo TextGrid con anotación fonética}, label={file:text_grid}]{archivos/03_01_text_grid_ejemplo.txt}



\section{Anotación manual a nivel de sentencia}

También se realizó una anotación a nivel de declaración de los audiolibros de Librivox. Para ello se seleccionaron audios correspondiendo al 10\% del corpus, es decir, 6 horas. Sobre estos audios se realizó una segmentación manual basada en símbolos de puntuación. 

Para la selección de los audio libros, se ordenaron las grabaciones en orden ascendente considerando el tamaño del audio original en segundos, para posteriormente seleccionar 3 horas de locutores de género femenino y 3 horas del género masculino, garantizando de esta manera un balance de género y también la maximización de locutores diferentes. Cada archivo será identificado con un prefijo F o M que significa el género del locutor, separado por una línea baja \_ y el consecutivo asignado.

Los archivos almacenados en Librivox usan el formato de compresión con pérdida MP3, para su tratamiento son transformados a formato sin compresión WAV utilizando el programa Sox \cite{Sox}.

Se realizó de igual manera una descarga manual de los textos correspondientes a los audio libros seleccionados, generando archivos de texto plano cuya primera línea es el título del texto. Las líneas siguientes corresponden al texto.

Como parte del pre-procesamiento del texto, se utilizaron expresiones regulares para segmentar los textos descargados de internet por símbolos de puntuación, y generando un nuevo archivo de texto plano donde cada línea contiene un índice, y la declaración correspondiente al segmento del texto.

Se muestra un ejemplo de texto tokenizado en el archivo \ref{file:texto_tokenizado}

\lstinputlisting[caption={Texto tokenizado}, label={file:texto_tokenizado}]{archivos/03_02_tokenizacion_texto.txt}

Los índices al comienzo de cada línea funcionan como identificadores en el proceso de anotación a nivel de declaración. Esto con el objetivo de facilitar el proceso de anotación al solo indicar en el campo de texto un índice en lugar del texto correspondiente. Posteriormente es posible reconstruir un archivo de anotación, reemplazando los índices por el texto correspondiente almacenado en el archivo de tokenización. El ejemplo de anotación de declaración del texto ejemplo se puede observar en el archivo \ref{file:text_grid_tokenizado}.

\lstinputlisting[caption={Segmento de Archivo TextGrid con anotación a nivel de declaración}, label={file:text_grid_tokenizado}]{archivos/03_03_text_grid_tokenizado.txt}


En adelante se denominará a esta anotación como corpus de pruebas.

% \section{Anotación automática a nivel de sentencia usando Libri Vox Spanish}

% \textcolor{red}{WIP}

% Esta sección corresponde a la generación de TextGrids usando los archivos fuentes descargados directamente desde librivox y comparándolos con la segmentación manual de \cite{LibriVox-Spanish}, de momento no tengo nada acá mas que experimentos fallidos usando Onsets y de pronto luego miro si hay tiempo de dejar este capítulo si encuentro alguna luz. De pronto usando DTW pasa algo acá pero no se.


\chapter{Segmentación y Alineación automática}

Como la creación de recursos de manera manual es costosa, y los requerimientos de volúmen de información son altos, la generación automática o semi automática de recursos de habla toma gran relevancia en la investigación actual.

El procesamiento de grabaciones del dominio público para la creación de recursos para la investigación requiere la resolución de dos tareas concretas, que son la segmentación y la anotación definidas en el capítulo anterior.

\section{Segmentación basada en silencio}

Las segmentación basada en silencio es una respuesta natural al problema de cómo segmentar señales de voz, pues existe una correspondencia directa entre símbolos de puntuación y pausas en la voz. De igual manera en una señal de voz, los locutores requieren pausas para tomar aire, lo que permite que exista una cadencia en la señal y espacios donde realizar los cortes.

Para identificar el proceso de segmentación por silencios, es necesario entender cómo se almacenan las señales de voz digitalmente y cómo se puede procesar este tipo de datos.

La manera m\'as sencilla de representar digitalmente una señal de voz es haciendo el uso de la Modulación por Impulsos Codificados o PCM por sus siglas en inglés, donde por medio de un transductor análogo, se capturan las variaciones en la presión del aire y se registran como un rango de valores normalizado. El senso de la señal se hace periódicamente a una frecuencia determinada generando de esta manera una secuencia de valores que representa la señal. El proceso de representar la presión del aire en un segmento de tiempo se denomina cuantización, y para señales de audio se utilizan valores normalizados entre -127 y 128 representados por 8 bits a 16000 Hz de frecuencia.

En la figura \ref{img:pcm} se muestra la visualización de una grabación correspondiente a la palabra agua, extraída del corpus Open Speech Corpus de palabaras, en su longitud original y aumentada en un 500\% y 1000\% usando Audacity \cite{audacity}

\begin{figure}[H]
\caption{Representación visual de una señal de voz}
\label{img:pcm}
\includegraphics[width=\textwidth]{imagenes/04_01_pcm.png}
\end{figure}

Al hablar de silencio, nos referimos a ausencia total o parcial de sonido, lo cual se puede relacionar directamente con la intensidad de la señal en un segmento de tiempo.

Definiremos intensidad como la sumatoria de las energías en un periodo de tiempo  \cite{Jurafsky2000SpeechRecognition}    

\begin{equation}
\label{eq:energy}
I = \sum{x^2}  
\end{equation}

Esto nos daría un indicador de toda la señal, sin embargo, es útil realizar este análisis por pequeños segmentos del audio para identificar en conjunto cuáles son los puntos donde la intensidad baja representa un espacio de silencio. Estos segmentos por lo general se definen en espacios de 25ms solapados cada 10ms \cite{Jurafsky2000SpeechRecognition}, lo cual da la idea de una señal continua, donde cada segmento comparte información con el anterior. 

Utilizando esta idea es posible tomar cualquier señal de voz, segmentarla cada 25ms y encontrar los segmentos consecutivos donde la intensidad sea baja o cercana a cero y estos segmentos consecutivos representarían pausas de silencio. Algunas consideraciones a tomar al usar esta aproximación es que incluso en medio de las señales de voz, existen subsegmentos donde la intensidad es baja, por ejemplo en la pronunciación de consonantes plosivas, como la b, c, d, g, p y t existe una interrupción momentánea y completa del flujo de aire, causando momentáneamente segmentos de silencio.

Con base en las anotaciones del corpus de experimentación fonética se determinó que la duración promedio de cada consonante plosiva es inferior a los 100ms. También, extrayendo información del corpus de pruebas se determinó que los espacios de silencio son de aproximadamente 500ms. Con esta información se decidió utilizar segmentos 250ms de longitud y desplazamiento de 100ms. Se determinan como silencios los conjuntos de cinco o más segmentos consecutivos de intensidad baja.

El criterio de silencio seleccionado como baja intensidad de múltiples segmentos consecutivos plantea retos con respecto al valor concreto de energía para determinar silencio o señal. Considerando que los locutores no grabaron a un volúmen estándar y que la relación ruido señal varía en todas las grabaciones es distinta, se propusieron dos aproximaciones para determinar los segmentos de silencio, la primera normalizando los valores de la señal y definiendo un umbral fijo de intensidad en proporción al valor mas alto de la grabación, considerando valores de 0.3, 0.15 y 0.05 que representan umbrales de intensidad del 30\%, 15\%, y 5\%. También se experimentó utilizando algoritmos de agrupamiento con dos centroides iniciales en 0 y la intensidad máxima. Los resultados se presentan en la tabla \ref{tab:resultados_segmentacion_silencios}.

\begin{table}[H]
\centering
\caption{Resultados de la segmentación por silencios}
% \caption{Speech English Corpus}
\label{tab:resultados_segmentacion_silencios}
\begin{tabular}{|l|l|}
\textbf{Unbral} & \textbf{Precisión} \\ \hline
Fijo 30\%  & 57.01\% \\  \hline
Fija 15\% & 83.26\% \\  \hline
Fija 5\% & 69.05\% \\  \hline
Dinámico  & 74.53\% \\  \hline
\end{tabular}
\end{table}


\section{Alineación ingenua}

La alineación consiste en, después de tener una segmentación por silencios en las grabaciones y una segmentación por símbolos de puntuación en la transcripción, encontrar la relación que existe entre los segmentos de audio y los de tokens del texto.

La primera aproximación ingenua consiste en relacionar en orden los segmentos de voz y texto, asumiendo que los signos de puntuación tienen siempre una ocurrencia acústica representada como silencio, y por consiguiente, al segmentar el texto por signos de puntuación existe una relación uno a uno con cada segmento delimitado por silencios.

Para la evaluación de este modelo se usa el indicador WER, del inglés Word Error Rate, el cual utilizando parámetros de sustitución (S), eliminación (E) e inserción (I) de la distancia de edición Levenstein \cite{Levenshtein_SPD66} y la totalidad de palabras (N) se obtiene la relación mostrada en la ecuación \ref{eq:wer}

\begin{equation}
    \label{eq:wer}
    WER = \frac{S + E + I}{N}
\end{equation}

Si los valores esperados y calculados difieren en su totalidad, es posible que el WER tenga valores superiores a 1, lo que indica una alineación completamente equivocada.

Como el proceso de alineación busca encontrar la similaridad de las anotaciones en los segmentos, se define un Align Error Rate, compuesto de la suma ponderada de cada WER en un segmento por la cantidad de palabras esperadas en el segmento, como se muestra en la ecuación \ref{eq:aer}

% \macb{Esta m\'etrica penaliza mucho m\'as las eliminaciones, si un segmento deb\'ia tener 3 palabras, pero resultaron 6, el WER (que es al menos 3 puntos de eliminaci\'on) va a costar el doble. Sin embargo no hay una motivaci\'on clara de porqu\'e se deber\'ia penalizar m\'as el costo de eliminaci\'on. Si por el contrario solo hay una palabra cuando deber\'ian haber 3, el WER ser\'a s\'olo de 0.33. Mi sugerencia es manejar una ponderaci\'on m\'as general, ponderando la sumatoria de WER por cada segmento con el n\'umero total de palabras en todos los segmentos. Sugerencia: $AER = \frac{\sum{WER_i}}{\sum{N_i}}$}



\begin{equation}
    \label{eq:aer}
    AER = \sum_{i=0}^{n}{WER_i * \frac{W_i}{W}}
\end{equation}

Donde $n$ es el número de segmentos en la grabación, $W$ representa el número de palabras en la transcripción de la grabación, $W_i$ el numero de palabras en el respectivo segmento y $WER_i$ el WER del respectivo segmento.

Calculando el AER para la anotación ingenua se obtiene el siguiente resultado, expresado en valor porcentual.

\begin{equation}
    AER_{ingenuo} = 79.13
\end{equation}

En la figura \ref{img:aer_ingenuo} se muestra gráficamente los resultados de cada AER por grabación evaluada, donde se observa que en la mayoría de casos, los valores AER son altos, aunque en algunas grabaciones de corta duración las pausas corresponden a signos de puntuación, estos fenómenos no representa una regla.

\begin{figure}[H]
\caption{Resultados de la alineación ingénua}
% \macb{usar notaci\'on fon\'etica, por ejemplo /a/ en SAMPA, la letra A en SAMPA es una vocal del ingl\'es, igualmente con E.}
\label{img:aer_ingenuo}
\begin{center}
\includegraphics[width=0.7\textwidth]{imagenes/04_01_aer_ingenuo.png}
\end{center}
\end{figure}

%Donde se observa que al ser un valor muy alto, dando la idea que muchos segmentos se encuentran desalineados


\section{Alineación basada en duración de fonemas}

El problema con la alineación ingénua consiste en que las pausas no necesariamente corresponden siempre a un símbolo de puntuación, pues en caso de oraciones largas, el locutor debe respirar, y en oraciones muy cortas, el locutor omite la pausa para dar continuidad a la lectura.

Resaltando que existe igualmente un orden secuencial entre ambos segmentos, la aproximación basada en la duración de fonemas utiliza la duración promedio de fonemas extraída del corpus fonético de experimentación y reportada en la tabla \ref{tab:duracion_promedio_fonemas}. Con esta información y transformando las letras de cada token a su correspondiente fonema, se obtiene una duración estimada de cada token, la cual puede ser usada para alinear con mayor precisión.



\begin{table}[H]
\centering
\caption{Distribución fonética}
\label{tab:duracion_promedio_fonemas}
\begin{tabular}{|l|l|l|}
\textbf{Fonema} &\multicolumn{1}{|p{2.2cm}|}{\textbf{Duración promedio}} & \multicolumn{1}{|p{2.2cm}|}{\textbf{Desviación estándar}} \\ \hline
sil & 0.6619 & 0.3700 \\ \hline
a   & 0.1421 & 0.0566 \\ \hline
o   & 0.1487 & 0.0607 \\ \hline
e   & 0.1204 & 0.0416 \\ \hline
n   & 0.1090 & 0.0442 \\ \hline
i   & 0.1294 & 0.0404 \\ \hline
l   & 0.1125 & 0.0530 \\ \hline
s   & 0.1811 & 0.1059 \\ \hline
t   & 0.0784 & 0.0507 \\ \hline
d   & 0.0877 & 0.0536 \\ \hline
R   & 0.0999 & 0.0517 \\ \hline
b   & 0.0826 & 0.0610 \\ \hline
r   & 0.0750 & 0.0254 \\ \hline
p   & 0.0692 & 0.0649 \\ \hline
j   & 0.1038 & 0.0659 \\ \hline
u   & 0.1441 & 0.0507 \\ \hline
m   & 0.1183 & 0.0515 \\ \hline
k   & 0.1224 & 0.0904 \\ \hline
g   & 0.1016 & 0.0970 \\ \hline
f   & 0.1402 & 0.0569  \\ \hline
c   & 0.1032 & 0.0838  \\ \hline
y   & 0.1118 & 0.0044  \\ \hline
C   & 0.1274 & 0.0248  \\ \hline
N   & 0.0575 & 0  \\ \hline
S   & 0.0927 & 0  \\ \hline
\end{tabular}
\end{table}

Este problema entonces se simplifica a uno de optimización donde dados dos arreglos numéricos, donde el primero representa la duración en segundos de cada segmento y el segundo la duración estimada de cada token, se debe encontrar una relación entre los índices de cada arreglo que minimice la diferencia de segundos entre los segmentos correspondientes. 

Con dicha estrategia se obtiene un AER de 58.21\%.

En la imagen \ref{img:aer_duracion} se muestran los resultados individuales de cada grabación. De esta manera se observa un conjunto de grabaciones de baja duración, inferior a los 80 segundos en las cuales el AER es del 10.83\%, en las grabaciones con duración superior el AER es de 94.68\%, el cual representa el mismo fenómeno de la anotación ingénua, pues la elección erronea de un segmento propaga el error en los segmentos siguientes.

\begin{figure}[H]
\caption{Resultados de la alineación por duración de fonemas}
% \macb{usar notaci\'on fon\'etica, por ejemplo /a/ en SAMPA, la letra A en SAMPA es una vocal del ingl\'es, igualmente con E.}
\label{img:aer_duracion}
\begin{center}
\includegraphics[width=0.7\textwidth]{imagenes/04_03_aer_duracion_fonemas.png}
\end{center}
\end{figure}

%\begin{equation}
%    AER_{duracion fonemas} = 12.5
%\end{equation}

\section{Segmentación basada en información acústica}

Las grabaciones voz no son señales estacionarias, pero al segmentar en pequeños trozos de tamaño inferior al de la duración de los fonemas, estos segmentos cuando pertenecen a vocales y consonantes sonoras son estacionarios.

Si bien las características morfológicas de los locutores modifican su timbre y tono de voz, a nivel acústico, la dicción de cada fonema es similar. En las vocales, que son los fonemas con más energía, las partes de la anatomía responsables de diferenciar los sonidos son la apertura de la glotis, la posici\'on de la lengua y la apertura de la boca. Lo cual da la idea que una descomposición acústica podría identificar este fenómeno.

Al descomponer cualquier señal utilizando la transformada de Fourier (ver equación \ref{eq:fourier}), se obtienen señales raíz de cada onda compleja. Con las señales de voz, los coeficientes máximos  en el orden ascendente de la frecuencia se denominan formantes, y el formante 1 (F1) y el formante 2 (F2) representan efectivamente el fenómeno efectuado por las cuerdas vocales vibrando y la resonancia dada por la cavidad bucal. Los formantes son harmónicos, la captura de estos se da en señales de cualquier tipo, sea de voz u otro fenómeno acústico, como las señales generadas por instrumentos musicales. Todos los formantes F1,F2,F3, etc. corresponden al pulso excitativo de la glotis y la resonancia en el tracto vocal.

\begin{equation}
\label{eq:fourier}    
f(x) = \frac{1}{2} \, a_{0} + \sum_{n=1}^{\infty} \left[
   a_{n}\,\mathbf{cos} (n\,x) + b_{n} \,\mathbf{sin} (n\,x) \right]
\end{equation}

Los formantes teóricos para las cinco vocales del español se muestran en la tabla \ref{tab:formantes_teoricos} tomada de \cite{Bradlow1995}. Dichos formantes fueron calculados promediando la medición de formantes F1 y F2 de varios locutores con dialecto peninsular.

\begin{table}[H]
\centering
\caption{Formantes teóricos para vocales del español \cite{Bradlow1995}}
\label{tab:formantes_teoricos}
\begin{tabular}{|l|l|l|l|l|}
\hline
\textbf{Vocal} & \textbf{f1} & \textbf{f2} & \textbf{std f1} & \textbf{std f2} \\ \hline
a   & 638 & 36 & 1353 & 84 \\ \hline
e   & 458 & 42 & 1800 & 131 \\ \hline
i   & 286 & 6  & 2200 & 131 \\ \hline
o   & 460 & 19 & 1019 & 99 \\ \hline
u   & 300 & 20 & 992  & 121 \\ \hline

\end{tabular}
\end{table}

También se realizó una observación de las vocales calculando sus formantes basados en la información extra\'ida de las palabras anotadas fonéticamente. Esta información se encuentra en la tabla \ref{tab:formantes_observador}. Los formantes fueron calculados con Parselmouth \cite{parselmouth}.

\begin{table}[H]
\centering
\caption{Formantes observados para vocales del español}
\label{tab:formantes_observador}
\begin{tabular}{|l|l|l|l|l|}
\hline
\textbf{Vocal} & \textbf{f1} & \textbf{f2} & \textbf{std f1} & \textbf{std f2} \\ \hline
a   & 682.84 & 1347.66 & 73.77 & 93.04 \\ \hline
e   & 494.84 & 1619.97 & 55.49 & 132.84\\ \hline
i   & 382.33 & 1657.33 & 50.65 & 155.24 \\ \hline
o   & 506.24 & 1101.14 & 60.20 & 170.64\\ \hline
u   & 434.77 & 984.185 & 50.32 & 193.66\\ \hline

\end{tabular}
\end{table}

A partir de esta información de la tabla \ref{tab:formantes_observador} se realizaron experimentos buscando la precisión en la identificación de las vocales. Para ello se tomó la norma L1 \cite{manhatattan_distance} de los dos primeros formantes, definiendo rango de aceptación, entre los 50Hz y 500Hz; y utilizando la desviación estándar de cada vocal como rango de aceptación. Adicionalmente a esto, se entrenó un algoritmo de agrupación no supervizado, K-means, inicializando los centroides en los formantes observados en el corpus de experimentación fonética.

Estos experimentos se pusieron a prueba en todo el corpus de experimentación fonética, de donde se extrajeron los segmentos correspondientes a vocales y se evaluó la precisión de cada aproximación.

Los resultados se reportan en la tabla \ref{tab:resultados_vocales}.

\begin{table}[H]
\centering
\caption{Precisión identificación vocálica}
\label{tab:resultados_vocales}
\begin{tabular}{|l|l|l|l|l|}
\hline
Aproximación              & Precisión & Recall & F1 Score \\ \hline
Teórica mas 50 Hz         & 44.36     & 45.51  & 44.93    \\ \hline
Teórica mas 100 Hz        & 46.29     & 47.49  & 46.88    \\ \hline
Teórica mas 200 Hz        & \textbf{50.36}     & 51.66  & 51.00    \\ \hline
Teórica mas 300 Hz        & 47.71     & 48.95  & 48.32    \\ \hline
Teórica mas 400 Hz        & 40.75     & 41.80  & 41.27    \\ \hline
Teórica mas 500 Hz        & 36.81     & 37.76  & 37.28    \\ \hline
Observados mas desviación & 41.76     & 42.84  & 42.29    \\ \hline
K-means mas 200           & 45.86     & 47.05  & 4645     \\ \hline
\end{tabular}

\end{table}

Para entender los bajos resultados presentados por la aproximación basada en los formantes observados y su desviación (50.36\% de precisión), se realizó un análisis visual sobre grabaciones con las vocales que se extrajeron del Open Speech Corpus. Se graficó simultaneamente la onda, los formantes F1 y F2 donde la intensidad superaba los 60 decibeles, es decir, $I > 60 dB$, y los formantes F1 y F2 promedio y sus desviaciones observados en el corpus. Las figuras \ref{img:formantes_a}, \ref{img:formantes_e}, \ref{img:formantes_i}, \ref{img:formantes_o} y \ref{img:formantes_u} muestran ejemplos de dicho análisis. Los puntos indican los formantes calculados, F1 en azul y F2 en rojo, las líneas azul y roja representan los valores promedios para los formantes 1 y 2 respectivamente, con una franja del mismo color y en leve transparencia representando los rangos aceptados para cada formante considerando la desviación estándar, en verde la señal, y en amarillo la intensidad.

% (Y explicar las lineas rojas, amarilla, verde. Recomiendo que los formantes observados sean graficados con la desviaci\'on st\'andar en transparencia. Tambi\'en, los dos formantes observados deber\'ian tener color distinto entre ellos y posiblemente en la misma escala de los calculados.)
%\ref{img:formantes_a} y e en la figura \ref{img:formantes_e}

% Ejemplo de cómo graficar la desviación estándar: https://i.stack.imgur.com/pDYIh.png


\begin{figure}[H]
\caption{Formantes de vocal A}
% \macb{usar notaci\'on fon\'etica, por ejemplo /a/ en SAMPA, la letra A en SAMPA es una vocal del ingl\'es, igualmente con E.}
\label{img:formantes_a}
\begin{center}
\includegraphics[width=0.7\textwidth]{imagenes/04_02_a.png}
\end{center}
\end{figure}

\begin{figure}[H]
\caption{Formantes de vocal E}
\label{img:formantes_e}
\begin{center}
\includegraphics[width=0.6\textwidth]{imagenes/04_02_e.png}
\end{center}
\end{figure}

\begin{figure}[H]
\caption{Formantes de vocal I}
\label{img:formantes_i}
\begin{center}
\includegraphics[width=0.6\textwidth]{imagenes/04_02_i.png}
\end{center}
\end{figure}

\begin{figure}[H]
\caption{Formantes de vocal O}
\label{img:formantes_o}
\begin{center}
\includegraphics[width=0.6\textwidth]{imagenes/04_02_o.png}
\end{center}
\end{figure}

\begin{figure}[H]
\caption{Formantes de vocal U}
\label{img:formantes_u}
\begin{center}
\includegraphics[width=0.7\textwidth]{imagenes/04_02_u.png}
\end{center}
\end{figure}

% Esto debería estar arriba.
% En ambas figuras se muestra en verde la onda, en azul el formante 1 en rojo el formante dos, en amarillo la intensidad cuyo valor se encuentra en el eje derecho y en rojo y azul claro los rangos esperados siendo la línea centra el formante observado promedio y las lineas adyacentes los valores de sumar y restar la desviación estándar.

En las gráficas se observa que en los rangos donde hay mas energía, los formantes se estabilizan cerca de ciertos valores, sin embargo no siempre el valor de estabilización corresponde a los rangos cercanos al formante promedio más o menos su desviación estándar.

Considerando que en los valores de los formantes y su rango de aceptación no genera los resultados esperados, se realizan experimentos con árboles de decisión. Para estos experimentos, se tomaron los mismos valores de los formantes y la intensidad usando también valores de los formantes 3, 4 y 5, que enriquecen la información acústica extraída de las grabaciones.

Los conjuntos de entrenamiento y pruebas se extrajeron del corpus de experimentación fonética, realizando una partición de 90\% para entrenamiento y 10\% para pruebas. Esta segmentación se realizó creando ventanas de tiempo de 60ms. Se contó la ocurrencia de cada fonema en las ventanas y sobre el total de apariciones de fonemas en las ventanas se determinó la proporción 90\%-10\% para los conjuntos de entrenamiento y pruebas.

Como los árboles de decisión son sensibles a la cantidad de grupos a clasificar, se realizaron experimentos transformando la anotación fonética de cada grabación. Las transformaciones usadas se muestran en la tabla \ref{tab:transformacion_fonetica_arboles}.

\begin{table}[H]
\centering
\caption{Clasificaci\'on fonética para modelos de \'arboles de decisiones.}
\label{tab:transformacion_fonetica_arboles}
\begin{tabular}{|l|l|l|l|l|}
\hline
\textbf{Fonema} & \multicolumn{1}{|p{2.5cm}|}{\textbf{Vocales / No Vocales}} &  \multicolumn{1}{|p{3cm}|}{\textbf{Consonantes sonoras / No sonoras}} &  \multicolumn{1}{|p{2.5cm}|}{\textbf{Tipos de articulación}} & \multicolumn{1}{|p{2.3cm}|}{\textbf{Vocales Extremas}} \\\hline
a   & a         & a          & a           & a         \\ \hline
o   & o         & o          & o           & no\_vocal \\ \hline
e   & e         & e          & e           & no\_vocal \\ \hline
n   & no\_vocal & sonora     & nasal       & no\_vocal \\ \hline
i   & i         & i          & i           & i         \\ \hline
l   & no\_vocal & sonora     & aproximante & no\_vocal \\ \hline
s   & no\_vocal & sonora     & fricativa   & no\_vocal \\ \hline
t   & no\_vocal & no\_sonora & plosiva     & no\_vocal \\ \hline
d   & no\_vocal & no\_sonora & plosiva     & no\_vocal \\ \hline
R   & no\_vocal & sonora     & vibrante    & no\_vocal \\ \hline
b   & no\_vocal & no\_sonora & plosiva     & no\_vocal \\ \hline
r   & no\_vocal & no\_sonora & plosiva     & no\_vocal \\ \hline
p   & no\_vocal & no\_sonora & plosiva     & no\_vocal \\ \hline
j   & no\_vocal & sonora     & aproximante & no\_vocal \\ \hline
u   & u         & u          & u           & u         \\ \hline
m   & no\_vocal & sonora     & nasal       & no\_vocal \\ \hline
k   & no\_vocal & no\_sonora & plosiva     & no\_vocal \\ \hline
g   & no\_vocal & sonora     & plosiva     & no\_vocal \\ \hline
f   & no\_vocal & sonora     & fricativa   & no\_vocal \\ \hline
y   & no\_vocal & sonora     & fricativa   & no\_vocal \\ \hline
C   & no\_vocal & sonora     & fricativa   & no\_vocal \\ \hline
N   & no\_vocal & sonora     & nasal       & no\_vocal \\ \hline
S   & no\_vocal & sonora     & fricativa   & no\_vocal \\ \hline
\end{tabular}
\end{table}

Los resultados de los experimentos se muestran a continuación en la tabla \ref{tab:resultados_arboles}

\begin{table}[H]
\centering
\caption{Resultados de la clasificación fonética usando árboles de decisión}
\label{tab:resultados_arboles}
\begin{tabular}{|l|l|l|l|}
\hline
Aproximación                     & Precisión & Recall & F1 Score \\ \hline
Todos los fonemas                & 42.34     & 42.34  & 42.34    \\ \hline
Tipos de articulación            & 50.76     & 52.30  & 52.30    \\ \hline
Consonantes sonoras / No sonoras & 62.86     & 62.86  & 62.86    \\ \hline
Vocales \ No Vocales             & 69.62     & 69.62  & 69.62    \\ \hline
Vocales extremas                 & \textbf{81.21}     & \textbf{81.21}  & \textbf{81.21}    \\ \hline


\end{tabular}
\end{table}

Utilizando el clasificador de vocales extremas, se realizó el cálculo del AER, haciendo una transformación de los tokens al diccionario fonético y posteriormente evaluando el WER del texto esperado con respecto al predicho por el clasificador.

En la imagen \ref{img:decision_tree} se visualiza el árbol de decisión usado para obtener los resultados del experimento, donde se tiene un árbol de 7 niveles, las características usadas para la separación de este árbol fueron las 5 primeras formantes y la intensidad.

\begin{figure}[H]
\caption{Árbol de decisión de vocales externas}
\label{img:decision_tree}
\begin{center}
\includegraphics[width=1.1\textwidth,angle=90]{imagenes/external_vowels.png}
\end{center}
\end{figure}

Como el clasificador retorna un posible fonema por cada segmento de duración de 60ms, se comprimió el resultado predicho agrupando todos los fonemas adyacentes en uno solo. Considerando de igual manera que la duración promedio de las vocales evaluadas es cercana a los 150ms con desviación de 50ms, los grupos con menos de 4 fonemas adyacentes se eliminan, pues no representan ninguna vocal.

Realizando estas adaptaciones, el valor de AER de la anotación usando información fonética fue: 89.56\% y los resultados individuales se muestran en la figura \ref{img:aer_vocales_externas}.

\begin{figure}[H]
\caption{Resultados de la alineación usando información fonética}
\label{img:aer_vocales_externas}
\begin{center}
\includegraphics[width=0.7\textwidth]{imagenes/04_04_aer_vocales_externas.png}
\end{center}
\end{figure}

El principal problema con esta aproximación es que, al transformar el texto a esta representación fonética, toda la información de las consonantes y dos vocales se pierde, reduciendo el tamaño del texto y haciendo que cada error le dé un valor más alto a cada WER local. En algunos casos, al finalizar la transformación del texto, se encontraron escenarios donde en un token no existía ninguna vocal externa, dejando este segmento de varios segundos como un solo fonema. Adicional a esto, la propagación de los errores a segmentos adyacentes sigue presente.

\section{Comparación de técnicas}

En este capítulo de discutieron tres aproximaciones para realizar una anotación automática de corpus extraídos de audiolibros, usando una segmentación automática basada en silencios y un umbral de ruido del 15\%, el cual da una precisión del 83.26\% la cual hace que sea bastante atractiva como herramienta de segmentación en comparación con una segmentación manual.

Con respecto a la anotación, los mejores resultados se obtuvieron a partir de la segmentación basada en duración promedio de fonemas. Para el caso puntual de los audiolibros donde la lectura es continua y hay una cadencia constante, esta técnica obtuvo buenos resultados, sin embargo, en el corpus escogido, existen distintos tipos de textos, y tipos de lectura, y textos donde el locutor enfatice mas las pausas, como en los textos líricos, esta aproximación basada en duración promedio de fonemas empeora los resultados. En casos, donde los textos son cortos y los locutores no tienen espacios para variar su cadencia, esta técnica muestra mejores resultados

La anotación basada en fonemas, aunque presenta una alta precisión a la hora de identificar características aisladas, no puede transferir esta precisión a la anotación del texto completo. Usando el clasificador con mejores resultados, que contempla solo vocales externas y no vocales, existen pequeños espacios de tiempo, donde las características no son estables y se genera un cambio de unidad fonética, que afecta los resultados. De igual manera, palabras que incluyen fonemas adyacentes con la misma categoría, como consonantes adyacentes o hiatos, deforman las palabras a un punto donde no hay información suficiente para determinar los límites en el tiempo.

Con estos resultados, el corpus generado por la técnica basada en fonemas, que tuvo un mejor AER, presentaba muchos errores. Realizando verificaciones manuales de los segmentos generados para grabaciones de mas de 100 segundos, se observaron desalineaciones que se propagaban hasta el final del texto, haciendo que los resultados no fueran apropiados para su publicación, pues el error es muy alto.

En comparación con herramientas existentes y abiertas, las técnicas exploradas en este trabajo no ofrecen ninguna ventaja. 

Es válido aclarar que muchas de las herramientas en el estado del arte utilizan modelos acústicos pre-generados y que son aveces usados en tareas de reconocimiento de voz. Es usual encontrar modelos que fueron generados por corpus que no son de licencia abierta, o no están disponibles, dando un poco de ventaja a la herramienta. Existen también herramientas de código abierto que usan herramientas pagas de terceros, y reportan mejores resultados. 



\addcontentsline{toc}{chapter}{Bibliography}
\bibliographystyle{apalike}
% \setcitestyle{authoryear,open={((},close={))}}
% \bibliographystyle{apalike}
% \bibliographystyle{apacite}
% \bibliography{bibliography/bibliography}
\bibliography{custom_references.bib,references.bib}



\end{document}
