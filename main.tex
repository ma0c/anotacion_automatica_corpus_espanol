\documentclass[a4paper,12pt,twoside]{report}
\usepackage[left=2cm,right=2cm,top=2cm,bottom=3cm]{geometry}
\usepackage[spanish]{babel}
\usepackage[utf8]{inputenc}
\usepackage{svg}
\usepackage{xcolor}
\usepackage{graphicx}

\title{Anotación automática de un corpus hablado de licencia abierta y larga duración para el lenguaje Español}
\author{Mauricio Collazos}
\date{March 2021}

\begin{document}

\maketitle

\begin{abstract}
    Los recursos abiertos para el procesamiento, reconocimiento, generación y demás tareas relacionadas con las tecnologías del lenguaje hablado son parte fundamental del proceso académico e investigativo. Estos recursos deben contener características muy específicas para la ejecución adecuada de la tarea relacionada, donde los niveles de anotación, duración de las grabaciones, variedad de locutores, balance de género, tamaño del vocabulario, relación de señal/ruido y conjunto de datos para pruebas son fundamentales. A pesar de existir diversos recursos disponibles para realizar las tareas mencionadas, los recursos abiertos para trabajar las tecnologías del hablada en la lengua española son escasos y en muchos casos insuficientes para realizar comparaciones con investigaciones del estado del arte propuestas para otros lenguajes. Este trabajo realiza una exploración de los recursos abiertos para el español y propone mecanismos para segmentar audio libros publicados por voluntarios bajo licencias abiertas en recursos apropiados para el procesamiento de voz.
\end{abstract}


\chapter{Introducción}

\section{Definición del problema}

Los recursos para la investigación de tecnologías del habla son de vital importancia para la generación y validación de nuevo conocimiento en el área. En la actualidad donde la investigación se ha decantado en su mayoría por el aprendizaje de máquina utilizando mecanismos de aprendizaje supervisado, semi supervisado y no supervisado \textcolor{red}{[CITA]}, la calidad de los datos usados para el entrenamiento, validación y pruebas definen en gran medida los resultados de la investigación.

Recursos representativos para realizar tareas comunes como el reconocimiento automático del habla y la síntesis del habla son TIMIT \textcolor{red}{[CITA]}, Switchboard \textcolor{red}{[CITA]}, Fisher \textcolor{red}{[CITA]} y Libri Speech \textcolor{red}{[CITA]}, recursos anotados desde el nivel fonético hasta el nivel de declaración \textcolor{red}{(acá estoy traduciendo utterance pero no me suena del todo)} altamente citados en investigaciones relacionadas con tecnologías del habla para la lengua inglesa.



\end{document}
